%%%%%%%%%%%%%%%%%%%%%%%%%%%%%%%%%%%%%%%%%%
% Engineering problems / LaTeX Template
%		Semester 6
%		Institut d'Optique Graduate School
%%%%%%%%%%%%%%%%%%%%%%%%%%%%%%%%%%%%%%%%%%
%	6N-IntNum-BlocRobot	/ Embedded System
%%%%%%%%%%%%%%%%%%%%%%%%%%%%%%%%%%%%%%%%%%
%
% Created by:
%	Julien VILLEMEJANE - 19/oct/2024	
%
%%%%%%%%%%%%%%%%%%%%%%%%%%%%%%%%%%%%%%%%%%
% Professional Newsletter Template
% LaTeX Template
% Version 1.0 (09/03/14)
%
% Created by:
% Bob Kerstetter (https://www.tug.org/texshowcase/) and extensively modified by:
% Vel (vel@latextemplates.com)
% 
% This template has been downloaded from:
% http://www.LaTeXTemplates.com
%
% License:
% CC BY-NC-SA 3.0 (http://creativecommons.org/licenses/by-nc-sa/3.0/)
%
%%%%%%%%%%%%%%%%%%%%%%%%%%%%%%%%%%%%%%%%%

\documentclass[a4paper,11pt,titlepage]{article} % The default font size is 10pt; 11pt and 12pt are alternatives

%%%%%%%%%%%%%%%%%%%%%%%%%%%%%%%%%%%%%%%%%%%%%%%%%%%%%%%%%%%%%%%%%%%%%%%%%%%%%%%%%%%%%%%%%%%%%%%%%%%%%%%%%%%%%%%%%%%%%%%%%%%%%%%%%%%%%%%%%%%%%%%%%%%%%%%%%%%%%%%%%%%%%%%%%%%%%%%%%%%%%%%%%%%%%%%%%%%%%%%%%%%%%%%%%%%%%%%%%%%%%%%%%%%%%%%%%%%%%%%%%%%%%%%%%%%%
\usepackage{opto_elec_villemejane}

%%%%%%%%%%%%%%%%%%%%%%%%%%%%%%%%%%%%%%%%%%%%%%%%
%%%%%%%%%%%%%%%%%%%%%%%%%%%%%%%%%%%%%%%%%%%%%%%%
\begin{document}



% Page de garde
\begin{titlepage}

\begin{center}
	\begin{minipage}{2.5cm}
	\begin{center}
		\includegraphics[width=8cm]{images/Logo-LEnsE.png}
	\end{center}
\end{minipage}\hfill
\begin{minipage}{10cm}
	\begin{center}
	\textbf{Institut d'Optique Graduate School }\\[0.1cm]
    \textbf{Interfaçage Numérique}


	\end{center}
\end{minipage}\hfill


\vspace{4cm}


{\huge \bfseries \textsc{Interfaçage Numérique}} \\[0.5cm]
{\large \bfseries Travaux Pratiques} \\[0.2cm]
Semestre 6

\vspace{2cm}
% Title
\rule{\linewidth}{0.3mm} \\[0.4cm]
{ \huge \bfseries\color{violet_iogs} Banc de mesure de rayonnement \\[0.4cm] }
\rule{\linewidth}{0.3mm} \\[1cm]

4 séances

\bigskip

\begin{center}
%	\includegraphics[width=0.5\textwidth]{images/image_gui.png}
\end{center}

\vfill

\textit{Ce sujet est disponible au format électronique sur le site du LEnsE - https://lense.institutoptique.fr/ dans la rubrique Année / Première Année / Interfaçage Numérique S6 / Banc de mesure de rayonnement lumineux.}

% Bottom of the page
%{\textbf{\large {Année universitaire} 2024-2025}}

\end{center}
\end{titlepage}

\newpage
\strut % empty page


%%%%%%%%%%%%%%%%%%%%%%%%%%%%%%%%%%%%%%%%%%%%%%%%
%%%%%%%%%%%%%    Intro
\newpage
\pagestyle{empty}

\begin{minipage}[c]{.25\linewidth}
	\includegraphics[width=5cm]{images/Logo-LEnsE.png}
\end{minipage} \hfill
\begin{minipage}[c]{.4\linewidth}

\begin{center}
\vspace{0.3cm}
{\Large \textsc{Interfaçage Numérique}}

\medskip

6N-047-SCI \qquad \textbf{\large Bloc Rayonnement}

\end{center}
\end{minipage}\hfill

\vspace{0.5cm}

\noindent \rule{\linewidth}{1pt}

{\noindent\Large  \rule[-7pt]{0pt}{30pt} \textbf{Banc de mesure de rayonnement}}

\noindent \rule{\linewidth}{1pt}

\bigskip 

%%%%%%%%%%%%%%%%%%%%%%%%%%%%%%%%%%%%%%%%%%%%%%%%
%%%%%%%%%%%%%    A A V

{\large À l'issue des séances de TP concernant le \textbf{bloc d'acquisition d'un diagramme de rayonnement}, les étudiant$\cdot$es seront capables de :}

\medskip

\begin{itemize}
	\item Développer et mettre en \oe{}uvre une \textbf{solution d'électronique embarquée} pour \textbf{acquérir des données analogiques} et commander un élément mobile.
	\item Mettre en \oe{}uvre un \textbf{protocole simple de communication} entre un ordinateur et un microcontrôleur pour transmettre des commandes et lire des données
	\item Optimiser une \textbf{interface informatique} de pilotage et d'affichage de données
\end{itemize}

\noindent \rule{\linewidth}{1pt}


%%%%%%%%%%%%%%%%%%%%%%%%%%%%%%%%%%%%%%%%%%%%%%%%
%%%%%%%%%%%%%    Objectifs

\section{Objectifs du mini-projet}

L'objectif principal de ce mini-projet est d'\textbf{automatiser un banc de mesure de rayonnement lumineux} à l'aide d'un ordinateur et d'une interface en Python. Le matériel est piloté par une carte à microcontroleur à laquelle il faudra envoyer des commandes selon un protocole série.

\medskip

Un diagramme de rayonnement lumineux est la \textbf{représentation graphique} de la \textbf{distribution angulaire} d'une grandeur caractérisant le rayonnement d'une source lumineuse.

\begin{center}
	\includegraphics[width=0.7\textwidth]{images/osram_sfh41747.png}
	
	Exemple de diagramme de rayonnement d'une LED OSRAM SFH 41747 (documentation technique)
\end{center}


\medskip

Vous aurez à votre disposition une \textbf{maquette} permettant la mise en mouvement d'un système de photodétection autour d'une source de puissance à LED. Cette maquette est pilotée par une carte Nucléo (contenant un microcontroleur).


%%%%%%%%%%%%%%%%%%%%%%%%%%%%%%%%%%%%%%%%%%%%%%%%
%%%%%%%%%%%%%    Déroulement

\section{Déroulement du bloc}

\textit{La liste des étapes à suivre pour la réalisation du programme embarqué de la plateforme robotique est donnée à titre indicatif. L'ordre et le choix des différentes étapes sont laissés à l'appréciation des différents binômes.}

\textit{Afin de faciliter la réutilisation des codes, il pourra être intéressant de définir des fonctions pour le pilotage des différents éléments.}

\subsection{Séance 1 / Arduino et Nucléo-STM32}

	\begin{description}
		\item[Etape 0 - 30 min] Installer les drivers STM32 et tester un premier programme
		\item[Etape 1 - 45 min] Piloter des sorties numériques - LED
		\item[Etape 2 - 45 min] Acquérir des données numériques - Bouton-poussoirs
		\item[Etape 3 - 45 min] Mettre en \oe{}uvre des interruptions sur des événements externes
		\item[Etape 4 - 45 min] Utiliser des sorties modulées en largeur d'impulsion (PWM) - LEDs
		\item[Etape 5 - 60 min] Acquérir des données analogiques - Potentiomètre
	\end{description}	

\subsection{Séance 2 / Prise en main de la maquette et automatisation de la mesure}

	\begin{description}
		\item[Etape 6 - 60 min] Piloter l'intensité des LEDs de la maquette
		\item[Etape 7 - 60 min] Piloter le servomoteur de la maquette
		\item[Etape 8 - 90 min] Définir et tester une première structure de code permettant de piloter le servomoteur en faisant une acquisition de la luminosité à pas régulier
		\item[Etape 9 - 60 min] Récupérer les données à l'aide du moniteur Série du logiciel Arduino
	\end{description}
	
\subsection{Séance 3 / Mise en place d'un protocole de communication}

Lors de cette séance, vous serez amené à mettre en place un protocole de communication entre la carte embarquée et l'ordinateur, d'abord à l'aide d'un moniteur série puis à l'aide d'un script en Python.

	\begin{description}
		\item[Etape 1 - 90 min] Mettre en place un protocole de communication basé sur une liste de commandes et intégrer à la structure du code embarqué
		\item[Etape 2 - 90 min] Utiliser la bibliothèque pySerial en Python pour envoyer les commandes à la carte Arduino
		\item[Etape 3 - 90 min] Tester la communication entre la carte Arduino et le script Python pour afficher les données
	\end{description}

\medskip

L'ordinateur, maitre de la communication, devra envoyer des commandes à la carte embarquée, qui devra les exécuter puis acquitter du bon déroulement de l'opération.\

\textit{Un exemple de code pour la transmission des commandes A et D, à analyser et à tester, est fourni.}


\subsection{Séance 4 / Application complète}

Cette dernière séance sera consacrée à la finalisation des différents programmes : embarqué sur la carte Arduino pour la mesure automatique et sur l'ordinateur pour le pilotage de la maquette et l'affichage des données.

Selon le temps restant, il sera possible d'intégrer les fonctions écrites en Python dans une interface graphique, dont la structure de base est fournie.


\newpage
\strut % empty page
%%%%%%%%%%%%%%%%%%%%%%%%%%%%%%%%%%%%%%%%%%%%%%%%
%%%%%%%%%%%%%    Séance 1 détaillée

\begin{minipage}[c]{.25\linewidth}
	\includegraphics[width=4cm]{images/Logo-LEnsE.png}
\end{minipage} \hfill
\begin{minipage}[c]{.4\linewidth}

\begin{center}
\vspace{0.3cm}
{\Large \textsc{Interfaçage Numérique}}

\medskip

6N-047-SCI \qquad \textbf{\Large Bloc Rayonnement}

\end{center}
\end{minipage}\hfill

\vspace{0.5cm}

\noindent \rule{\linewidth}{1pt}

{\noindent\Large \rule[-7pt]{0pt}{30pt} Arduino et Nucléo-STM32} 

\noindent \rule{\linewidth}{1pt}


%%%%%%%%%%%%%%%%%%%%%%%%%%%%%%%%%%%%%%%%%%%%%%%%
%%%%%%%%%%%%%    Objectifs
\section{Objectifs de la séance}

Cette première séance est consacrée à la découverte de la \textbf{programmation de systèmes embarqués} (ici des cartes \textbf{Nucléo} de \textit{STMicroelectronics}, basées sur des microcontroleurs \textit{STM32}) et la prise en main de l'interface de développement \textbf{Arduino}.


%%%%%%%%%%%%%%%%%%%%%%%%%%%%%%%%%%%%%%%%%%%%%%%%
%%%%%%%%%%%%%    Arduino
\subsection{IDE Arduino}

\textbf{Arduino} est une plateforme open-source utilisée pour créer des projets électroniques. Elle est composée de deux éléments principaux : \textbf{une carte matérielle} (contenant un microcontrôleur) et \textbf{un environnement de développement} (IDE Arduino) qui permet de programmer la carte.

\begin{center}
	\includegraphics[width=0.15\textwidth]{images/Arduino_Logo.png}
\end{center}

Nous nous intéresserons ici qu'à l'environnement de développement qui, après installation d'une extension, permet de programmer d'autres cartes à microcontrôleurs.


%%%%%%%%%%%%%%%%%%%%%%%%%%%%%%%%%%%%%%%%%%%%%%%%
%%%%%%%%%%%%%    Nucleo
\subsection{Carte Nucleo-STM32}

Les cartes Nucleo sont des \textbf{plateformes de développement} basées sur les \textbf{microcontrôleurs STM32} de \textit{STMicroelectronics}. Elles sont conçues pour faciliter le prototypage et le développement de projets embarqués, similaires aux cartes Arduino, mais elles sont souvent utilisées pour des applications plus complexes et performantes.


\begin{center}
	\includegraphics[width=0.3\textwidth]{images/nucleo_board.jpg}
\end{center}

Elles sont équipées d'un débogueur ST-LINK intégré, ce qui permet de programmer et de déboguer le microcontrôleur directement sans matériel additionnel.

\textsl{Le brochage de la carte Nucleo L476RG est fournie en annexe à ce document} : \hyperref[doc:nucleo_pins_476RG]{Brochage Nucléo L476RG}


%%%%%%%%%%%%%%%%%%%%%%%%%%%%%%%%%%%%%%%%%%%%%%%%
%%%%%%%%%%%%%    Installation des cartes STM32
\section{Etape 0a / Installation des cartes STM32}

L'interface de développement \textbf{Arduino}, ainsi que les bibliothèques associées, est populaire pour les projets \textit{Do It Yourself}, l'éducation et le prototypage rapide en raison de sa simplicité et de son accessibilité. Cependant, elle est initialement prévue pour programmer des cartes de type \textbf{Arduino}.

Pour pouvoir bénéficier de l'environnement \textbf{Arduino} pour \textbf{d'autres cartes de prototypage}, il est indispensable d'installer les extensions associées à ces autres cartes.

\textbf{Attention !} La version 2 de l'IDE Arduino est fortement conseillée pour bénéficier des dernières évolutions du langage et de l'interface, ainsi que pour garantir une pleine compatibilité avec les cartes Nucléo.

\subsection{Support des cartes STM32}

Avant de pouvoir utiliser l'environnement Arduino pour programmer des cartes intégrant des microcontrôleurs de type STM32, il faut installer le \textbf{support pour ces microcontrôleurs}.

Dans le menu \textsc{\textbf{Fichiers} / \textbf{Préférences}}, sélectionner le volet \textsc{\textbf{Paramètres}}.

\begin{center}
	\includegraphics[width=0.75\textwidth]{images/arduino_preferences.png}
\end{center}

Dans la fenêtre \textsc{URL de gestionnaire de cartes supplémentaires}, ajouter l'adresse suivante :

\href{https://GitHub.com/stm32duino/BoardManagerFiles/raw/main/package_stmicroelectronics_index.json}{https://GitHub.com/stm32duino/BoardManagerFiles/raw/main/package\_stmicroelectronics\_index.json}

\begin{center}
	\includegraphics[width=0.6\textwidth]{images/arduino_preferences_url.png}
\end{center}

\subsection{Extension STM32 MCU based boards}

Il faut ensuite télécharger les \textbf{bibliothèques} liées aux cartes intégrant des \textbf{microcontrôleurs STM32} de \textit{STMicroelectronics}.

Aller dans le menu \textsc{\textbf{Outils} / \textbf{Carte} / \textbf{Gestionnaire de carte}}. 

\begin{center}
	\includegraphics[width=0.3\textwidth]{images/arduino_gestion_cartes.png}
\end{center}

Dans la partie droite de l'interface Arduino, un volet \textsc{\textbf{Gestionnaire de carte}} s'ouvre. Dans la zone de recherche, taper STM32.

Dans la liste, installer alors l'extension : \textbf{STM32 based boards par STMicroelectronics}.



\bigskip 

\textit{Vous trouverez également des ressources concernant les \textbf{microcontrôleurs} et les \textbf{systèmes embarqués} à l'adresse suivante :}

\href{https://iogs-lense-training.github.io/nucleo-basics/contents/general.html}{https://iogs-lense-training.github.io/nucleo-basics/contents/general.html}


\cleardoublepage
%%%%%%%%%%%%%%%%%%%%%%%%%%%%%%%%%%%%%%%%%%%%%%%%
%%%%%%%%%%%%%    Premier programme
\section{Etape 0b / Tester un premier programme}

Afin de vérifier que toute la chaîne de prototypage est opérationnelle, nous allons nous intéresser à un \textbf{programme de base} permettant de faire \textbf{clignoter une LED} présente par défaut sur la carte Nucléo (c'est également vrai sur les cartes Arduino).

Sélectionner \textsc{\textbf{Fichier} / \textbf{Exemples} / \textbf{01.Basics} / \textbf{Blink}} dans la barre de menu.

Le programme ressemble à celui-ci :

\begin{lstlisting}
void setup() {
  // initialize digital pin LED_BUILTIN as an output.
  pinMode(LED_BUILTIN, OUTPUT);
}

void loop() {
  digitalWrite(LED_BUILTIN, HIGH);  
  	// turn the LED on (HIGH is the voltage level)
  delay(1000);                      
  	// wait for a second
  digitalWrite(LED_BUILTIN, LOW);   
  	// turn the LED off by making the voltage LOW
  delay(1000);                      
  	// wait for a second
}
\end{lstlisting}

\textit{Sur la carte Nucléo la sortie LED\_BUILTIN correspond à la LED nommée LD2.}

Le langage utilisé par l'environnement Arduino est un langage proche du C++. Le programme ainsi écrit doit nécessairement \textbf{être compilé} avant de pouvoir \textbf{être téléversé} sur la carte où il sera ensuite \textbf{exécuté}.

\subsection{Structure du code}

Un programme Arduino (comme tout autre programme embarqué) est constitué de \textbf{deux étapes principales} : 

\begin{itemize}
	\item une \textbf{initialisation} (fonction \textsl{setup()} pour Arduino) : exécutée une fois à la mise sous tension de la carte ou lors de l'appui sur le bouton Reset.
	\item une \textbf{boucle infinie} (fonction \textsl{loop()} pour Arduino) : exécutée de manière infinie. Cette boucle a pour principale mission, sur un système embarqué, de récupérer les valeurs des entrées, de calculer les valeurs des sorties et de mettre à jour les sorties.
\end{itemize}

\begin{center}
	\includegraphics[width=0.4\textwidth]{images/arduino_program_structure.png}
\end{center}

\textit{D'autres étapes sont possibles lorsqu'on autorise le fonctionnement par interruption (voir dans la suite de ce document).}



\subsection{Choix d'une carte}

La compilation d'un tel programme se fait pour une cible particulière. Avant de pouvoir compilé, il est donc nécessaire de préciser sur quel microcontrôleur (ou quelle carte de prototypage) ce code sera exécuté.

Pour cela, dans la barre de menu, sélectionner \textsc{\textbf{Outils} / \textbf{Cartes}}.

Dans le cas d'une carte Nucléo de type L476RG, sélectionner ensuite \textsc{\textbf{STM32 MCU based boards} / \textbf{Nucleo-64}}. Le format pourra changer s'il s'agit d'une autre carte.

\begin{center}
	\includegraphics[width=0.75\textwidth]{images/arduino_outils_cartes_nucleo64.png}
\end{center}

Puis, dans le menu \textsc{\textbf{Outils} / \textbf{Board part number}}, sélectionner \textbf{Nucleo L476RG}.

\begin{center}
	\includegraphics[width=0.75\textwidth]{images/arduino_cartes_nucleo_lxxx.png}
\end{center}

\subsection{Compilation}

Il est maintenant possible de compiler le programme. Pour cela, cliquer sur la première icône de la barre d'action :

\begin{center}
	\includegraphics[width=0.9\textwidth]{images/arduino_compile_run.png}
\end{center}

\bigskip


\subsection{Connexion à une carte Nucléo et téléversement}

Il faut ensuite connecter la carte en USB.

Dans la barre des actions possibles sous Arduino, sélectionner le port de communication sur lequel est connectée la carte Nucléo.

\begin{center}
	\includegraphics[width=0.4\textwidth]{images/arduino_cartes_nucleo_comXX.png}
\end{center}

Pour téléverser ensuite le code dans la carte, cliquer sur la seconde icône de la barre d'actions (en forme de flèche vers la droite).



\cleardoublepage
%%%%%%%%%%%%%%%%%%%%%%%%%%%%%%%%%%%%%%%%%%%%%%%%
%%%%%%%%%%%%%    Premier programme
\section{Etape 1 / Piloter des sorties numériques - LED}

Afin de pouvoir interagir avec le monde extérieur, les microcontrôleurs disposent d'un ensemble d'\textbf{entrées} et de \textbf{sorties}. 

Chacune de ces entrées-sorties portent un nom, au format \textsc{Px\_n}, où \textsc{x} est le nom du port (A, B...) et \textsc{n} le numéro de la broche.

\begin{center}
	\includegraphics{images/nucleo_pin_functions.png}
	
	Exemple de la broche PA\_7 (port A, broche 7)
\end{center}


\subsection{Choix d'une broche}

Toutes les broches peuvent être utilisées en \textbf{entrée} ou en \textbf{sortie} \textbf{numérique}, c'est à dire un type de signal qui ne peut prendre que \textbf{deux états} : haut ou bas (aussi appelés 1 ou 0, ou encore \textit{HIGH} et \textit{LOW} en Arduino). 

\textit{Certaines broches ont également d'autres fonctionnalités : entrées analogiques, sorties modulées PWM, communication série...}


\subsection{Câblage d'une LED}

Nous allons voir ici comment connecter une LED à la carte Nucléo sur la broche D10 par exemple de la carte (ou PB6 - port B, broche 6).

Les schémas de câblage possibles sont les suivants :

\begin{center}
	\includegraphics[width=0.8\textwidth]{images/MINE_Nucleo_LED_Connexion.png}
	
	Exemple de la broche D10 reliée à une LED (ou PB6 - port B, broche 6)
\end{center}

Dans le cas des cartes Nucléo, la tension $\operatorname{VDD}$ est égale à $3.3\operatorname{V}$.

Il est indispensable d'associer la LED à une \textbf{résistance de protection}, permettant de limiter le courant : 

$$R_{LED} > \frac{VDD - V_{F}}{I_{Fmax}}$$

\textit{Les valeurs $V_F$ et $I_{Fmax}$ dépendent de la LED utilisée et sont à chercher dans sa documentation technique.}



\subsection{Programme}

Pour le programme, il est possible de s'inspirer des exemples fournis dans le logiciel Arduino : \textsc{Fichier / Exemples}.

Par exemple, pour piloter une sortie numérique, on pourra utiliser le programme \textsc{01. Basics / Blink}.

\subsubsection{Paramétrage}

Pour \textbf{configurer une broche en sortie}, il faut ajouter l'instruction suivante dans la fonction \textsl{setup()} (où \textsl{LED1} est le nom d'une broche du composant) :

\begin{lstlisting}
pinMode(PB6, OUTPUT);
\end{lstlisting}

\subsubsection{Utilisation}

Pour \textbf{affecter une valeur à une broche en sortie}, il faut utiliser une des deux instructions suivantes (où \textsl{LED1} est le nom d'une broche du composant) selon que l'on veut mettre la sortie à l'état bas (\textit{LOW}) ou à l'état haut (\textit{HIGH}) :

\begin{lstlisting}
digitalWrite(PB6, LOW);
digitalWrite(PB6, HIGH);
\end{lstlisting}


\subsubsection{Utilisation de constantes}

Afin de simplifier la lecture du code, il est possible d'\textbf{attribuer un nom différent à votre broche} en affectant une \textbf{constante entière} à la valeur de la broche utilisée. Cette définition devra se faire en dehors de toute fonction, afin que la constante associée soit \textbf{globale}.

\begin{lstlisting}
const int led1 = PB6;
\end{lstlisting}

\medskip

Il sera alors possible d'utiliser cette constante dans le reste du programme :

\begin{lstlisting}
pinMode(led1, OUTPUT);
\end{lstlisting}

\begin{lstlisting}
digitalWrite(led1, LOW);
\end{lstlisting}




%%%%%%%%%%%%%%%%%%%%%%%%%%%%%%%%%%%%%%%%%%%%%%%%
%%%%%%%%%%%%%    Présentation de la maquette
\newpage
\begin{minipage}[c]{.25\linewidth}
	\includegraphics[width=4cm]{images/Logo-LEnsE.png}
\end{minipage} \hfill
\begin{minipage}[c]{.4\linewidth}

\begin{center}
\vspace{0.3cm}
{\Large \textsc{Interfaçage Numérique}}

\medskip

6N-047-SCI \qquad \textbf{\large Bloc Rayonnement}

\end{center}
\end{minipage}\hfill

\vspace{0.5cm}

\noindent \rule{\linewidth}{1pt}

{\noindent\Large \rule[-7pt]{0pt}{30pt} \textbf{Maquette Rayonnement} / Présentation du matériel}

\noindent \rule{\linewidth}{1pt}

\bigskip

{\LARGE La tension maximale admissible par le servomoteur est de $6\operatorname{V}$ !}

{\Large Le courant maximal admissible par la LED de puissance est de $200\operatorname{mA}$ !}

\bigskip

\noindent \rule{\linewidth}{1pt}






\subsubsection{Utilisation de la sortie modulée PB7}
	
\begin{lstlisting}
LL_GPIO_SetAFPin_0_7(GPIOB,  GPIO_PIN_7,  GPIO_AF1_TIM2);
\end{lstlisting}


\section{Alimentations des éléments}




\section{Brochage de la carte Nucléo}
\begin{center}

\begin{tabular}{|l|l|l|l|}
\hline 
Maquette & \textbf{Broche Nucléo} & Type & Description \\ 
\hline 
\textsc{LED1} & PC7 & Sortie / PWM & Led active à l'\textbf{état haut}\\ 
\textsc{LED2} & PB13 & Sortie / PWM & Led active à l'\textbf{état bas}\\ 
\hline 
\textsc{SW1} & PC6 & Entrée & Bouton-poussoir, par défaut état bas\\ 
\textsc{SW2} & PC8 & Entrée & Bouton-poussoir, par défaut état bas\\ 
\textsc{UserButton} & PC13 & Entrée & Bouton-poussoir, par défaut état haut\\
\hline 
\textsc{Servo} & PB7 & Sortie / PWM & Servomoteur - T = 20 ms\\ 
\hline 
\textsc{Photodiode} & PC3 & Entrée analogique & Photodiode \\
 & & & (montage simple avec R variable)\\ 
\hline 
\textsc{Résistance Thermique} & PC1 & Entrée analogique & Résistance Thermique CT10k\\ 
\hline 
\textsc{SPI} & & & \\ 
\textit{SCK} & PA5 & Sortie & Signal d'horloge\\
\textit{MISO} & PA6 & Entrée & Données entrantes\\
\textit{MOSI} & PA7 & Sortie & Données sortantes\\
\hline 
\textsc{Pot Num} & SPI & MCP4132 & Potentiomètre Numérique\\ 
 & & & Monté dans un pont diviseur\\ 
\textit{CS} & PB9 & Sortie & \\ 
\textit{Pot Num In} & PB0 & Entrée analogique & Sortie du pont diviseur\\ 
\hline 
\textsc{LED Puissance} & SPI & MCP4132 & Potentiomètre numérique / Contrôle\\ 
 & & & courant dans Driver BCR430-UW6\\ 
\textit{CS} & PB5 & Sortie & \\

\hline 
\end{tabular} 

\end{center}




\subsubsection{Liste des commandes - côté maitre}

\begin{center}

\begin{tabular}{|l|l|l|}
\hline 
\textbf{Commande PC} & Description & \textbf{Réponse Nucléo} \\ 
\hline 
\textsc{!T?} & Test de la transmission & \textsc{!T;} \\ 
\hline 
\textsc{!A:value?} & Transmission de l'angle de départ souhaité pour le diagramme & \textsc{!A:value;} \\ 
 &  valeur entière en degré &  \\
 & \textit{Si angle non compris entre -90 et 90} & \textsc{value}$ = -100$ \\ 
\hline 
\textsc{!B:value?} & Transmission de l'angle final souhaité pour le diagramme & \textsc{!B:value;} \\ 
 &  valeur entière en degré &  \\ 
 & \textit{Si angle non compris entre -90 et 90} & \textsc{value}$ = -100$ \\
\hline 
\textsc{!C:value?} & Transmission du pas angulaire souhaité pour le diagramme & \textsc{!C:value;} \\ 
 &  valeur entière en degré &  \\ 
 & \textit{Si angle non compris entre -90 et 90} & \textsc{value}$ = -100$ \\
\hline 
\textsc{!S?} & Lancement de l'acquisition & \textsc{!S:nb;} \\ 
 &  nb est le nombre d'échantillons qui seront acquis par la carte &  \\ 
 & \textit{Si les angles fournis sont non compatibles} & \textsc{nb}$ = 0$ \\
\hline 
\textsc{!E?} & Test de fin de l'acquisition & \textsc{!E:Y/N;} \\
 &  (Y)es or (N)o &  \\  
\hline 
\textsc{!D:index?} & Demande de récupération d'une donnée & \textsc{!D:index:value;} \\
 &  index correspond au numéro de l'échantillon souhaité &  \\  
 &  value correspond à la valeur de l'échantillon souhaité &  \\  
\hline 
\end{tabular} 

\end{center}


%%%%%%%%%%%%%%%%%%%%%%%%%%%%%%%%%%%%%%%%%%%%%%%%
%%% RESSOURCES COMPLEMENTAIRES		

\newpage
% Ressources
\begin{center}
	\begin{minipage}{2.5cm}
	\begin{center}
		\includegraphics[width=5cm]{images/Logo-LEnsE.png}
	\end{center}
\end{minipage}\hfill
\begin{minipage}{10cm}
	\begin{center}
	\textbf{Institut d'Optique Graduate School }\\[0.1cm]
    \textbf{Interfaçage Numérique}


	\end{center}
\end{minipage}\hfill


\vspace{2cm}


{\Large \bfseries \textsc{Interfaçage Numérique}} \\[0.5cm]
{\large \bfseries Travaux Pratiques} \\[0.2cm]
Semestre 6

\vspace{1cm}

% Title
\rule{\linewidth}{0.4mm} \\[0.4cm]
{ \Large \bfseries\color{violet_iogs} Ressources \\[0.4cm] }
\rule{\linewidth}{0.4mm} \\[1cm]
{\large Bloc Rayonnement}

\end{center}

\vspace{3cm}

\textbf{\large Liste des ressources}
\begin{itemize}
	\item \hyperref[doc:robot_schematic]{Schéma de la carte de la maquette de rayonnement}
	\item \hyperref[doc:robot_pcb]{PCB de la carte de la maquette de rayonnement}
\end{itemize}

\vfill

\newpage
\strut % empty page
% Ressources
\titleformat{\section}
  {\null}{}{0pt}{}


\includepdf[pages=1, landscape=true, pagecommand={\section{\texorpdfstring{\hspace{-1em}}{Schéma Carte Rayonnement}}}\label{doc:robot_schematic}]{ressources/Rayonnement_L476RG_sch.pdf}


\includepdf[pages=1, pagecommand={\section{\texorpdfstring{\hspace{-1em}}{PCB Carte Rayonnement}}}\label{doc:robot_pcb}]{ressources/Rayonnement_L476RG_pcb.pdf}

\end{document}


