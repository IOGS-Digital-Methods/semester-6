%%%%%%%%%%%%%%%%%%%%%%%%%%%%%%%%%%%%%%%%%%
% Engineering problems / LaTeX Template
%		Semester 6
%		Institut d'Optique Graduate School
%%%%%%%%%%%%%%%%%%%%%%%%%%%%%%%%%%%%%%%%%%
%	6N-IntNum-BlocRobot	/ Embedded System
%%%%%%%%%%%%%%%%%%%%%%%%%%%%%%%%%%%%%%%%%%
%
% Created by:
%	Julien VILLEMEJANE - 19/oct/2024	
%
%%%%%%%%%%%%%%%%%%%%%%%%%%%%%%%%%%%%%%%%%%
% Professional Newsletter Template
% LaTeX Template
% Version 1.0 (09/03/14)
%
% Created by:
% Bob Kerstetter (https://www.tug.org/texshowcase/) and extensively modified by:
% Vel (vel@latextemplates.com)
% 
% This template has been downloaded from:
% http://www.LaTeXTemplates.com
%
% License:
% CC BY-NC-SA 3.0 (http://creativecommons.org/licenses/by-nc-sa/3.0/)
%
%%%%%%%%%%%%%%%%%%%%%%%%%%%%%%%%%%%%%%%%%

\documentclass[a4paper,11pt,titlepage]{article} % The default font size is 10pt; 11pt and 12pt are alternatives

%%%%%%%%%%%%%%%%%%%%%%%%%%%%%%%%%%%%%%%%%%%%%%%%%%%%%%%%%%%%%%%%%%%%%%%%%%%%%%%%%%%%%%%%%%%%%%%%%%%%%%%%%%%%%%%%%%%%%%%%%%%%%%%%%%%%%%%%%%%%%%%%%%%%%%%%%%%%%%%%%%%%%%%%%%%%%%%%%%%%%%%%%%%%%%%%%%%%%%%%%%%%%%%%%%%%%%%%%%%%%%%%%%%%%%%%%%%%%%%%%%%%%%%%%%%%
\usepackage{opto_elec_villemejane}

%%%%%%%%%%%%%%%%%%%%%%%%%%%%%%%%%%%%%%%%%%%%%%%%
%%%%%%%%%%%%%%%%%%%%%%%%%%%%%%%%%%%%%%%%%%%%%%%%
\begin{document}



% Page de garde
\begin{titlepage}

\begin{center}
	\begin{minipage}{2.5cm}
	\begin{center}
		\includegraphics[width=8cm]{images/Logo-LEnsE.png}
	\end{center}
\end{minipage}\hfill
\begin{minipage}{10cm}
	\begin{center}
	\textbf{Institut d'Optique Graduate School }\\[0.1cm]
    \textbf{Interfaçage Numérique}


	\end{center}
\end{minipage}\hfill


\vspace{4cm}


{\huge \bfseries \textsc{Interfaçage Numérique}} \\[0.5cm]
{\large \bfseries Travaux Pratiques} \\[0.2cm]
Semestre 6

\vspace{2cm}
% Title
\rule{\linewidth}{0.3mm} \\[0.4cm]
{ \huge \bfseries\color{violet_iogs} Banc de mesure de rayonnement \\[0.4cm] }
\rule{\linewidth}{0.3mm} \\[1cm]

4 séances

\bigskip

\begin{center}
%	\includegraphics[width=0.5\textwidth]{images/image_gui.png}
\end{center}

\vfill

\textit{Ce sujet est disponible au format électronique sur le site du LEnsE - https://lense.institutoptique.fr/ dans la rubrique Année / Première Année / Interfaçage Numérique S6 / Banc de mesure de rayonnement lumineux.}

% Bottom of the page
%{\textbf{\large {Année universitaire} 2024-2025}}

\end{center}
\end{titlepage}

\newpage
\strut % empty page


%%%%%%%%%%%%%%%%%%%%%%%%%%%%%%%%%%%%%%%%%%%%%%%%
%%%%%%%%%%%%%    Intro
\newpage
\pagestyle{empty}

\begin{minipage}[c]{.25\linewidth}
	\includegraphics[width=5cm]{images/Logo-LEnsE.png}
\end{minipage} \hfill
\begin{minipage}[c]{.4\linewidth}

\begin{center}
\vspace{0.3cm}
{\Large \textsc{Interfaçage Numérique}}

\medskip

6N-047-SCI \qquad \textbf{\large Bloc Rayonnement}

\end{center}
\end{minipage}\hfill

\vspace{0.5cm}

\noindent \rule{\linewidth}{1pt}

{\noindent\Large  \rule[-7pt]{0pt}{30pt} \textbf{Banc de mesure de rayonnement}}

\noindent \rule{\linewidth}{1pt}

\bigskip 

%%%%%%%%%%%%%%%%%%%%%%%%%%%%%%%%%%%%%%%%%%%%%%%%
%%%%%%%%%%%%%    A A V

{\large À l'issue des séances de TP concernant le \textbf{bloc d'acquisition d'un diagramme de rayonnement}, les étudiant$\cdot$es seront capables de :}

\medskip

\begin{itemize}
	\item Développer et mettre en \oe{}uvre une \textbf{solution d'électronique embarquée} pour \textbf{acquérir des données analogiques} et commander un élément mobile.
	\item Mettre en \oe{}uvre un \textbf{protocole simple de communication} entre un ordinateur et un microcontrôleur pour transmettre des commandes et lire des données
	\item Optimiser une \textbf{interface informatique} de pilotage et d'affichage de données
\end{itemize}

\noindent \rule{\linewidth}{1pt}


%%%%%%%%%%%%%%%%%%%%%%%%%%%%%%%%%%%%%%%%%%%%%%%%
%%%%%%%%%%%%%    Objectifs

\section{Objectifs du mini-projet}

L'objectif principal de ce mini-projet est d'\textbf{automatiser un banc de mesure de rayonnement lumineux} à l'aide d'un ordinateur et d'une interface en Python. Le matériel est piloté par une carte à microcontroleur à laquelle il faudra envoyer des commandes selon un protocole série.

\medskip

Un diagramme de rayonnement lumineux est la \textbf{représentation graphique} de la \textbf{distribution angulaire} d'une grandeur caractérisant le rayonnement d'une source lumineuse.

\begin{center}
	\includegraphics[width=0.7\textwidth]{images/osram_sfh41747.png}
	
	Exemple de diagramme de rayonnement d'une LED OSRAM SFH 41747 (documentation technique)
\end{center}


\medskip

Vous aurez à votre disposition une \textbf{maquette} permettant la mise en mouvement d'un système de photodétection autour d'une source de puissance à LED. Cette maquette est pilotée par une carte Nucléo (contenant un microcontroleur).


%%%%%%%%%%%%%%%%%%%%%%%%%%%%%%%%%%%%%%%%%%%%%%%%
%%%%%%%%%%%%%    Déroulement

\section{Déroulement du bloc}

\textit{La liste des étapes à suivre pour la réalisation du programme embarqué de la plateforme de rayonnement lumineux est donnée à titre indicatif. L'ordre et le choix des différentes étapes sont laissés à l'appréciation des différents binômes.}

\textit{Afin de faciliter la réutilisation des codes, il pourra être intéressant de définir des fonctions pour le pilotage des différents éléments.}

\subsection{Séance 1 / Arduino et Nucléo-STM32 (sans maquette !!)}

\textit{Le sujet de cette séance est fourni dans un document annexe, disponible aussi sur le site du LEnsE - https://lense.institutoptique.fr/ dans la rubrique Année / Première Année / Interfaçage Numérique S6 / Bloc Systèmes embarqués / Intro Arduino et STM32.}

	\begin{description}
		\item[Etape 0 - 30 min] Installer les drivers STM32 et tester un premier programme
		\item[Etape 1 - 45 min] Piloter des sorties numériques - LED
		\item[Etape 2 - 45 min] Acquérir des données numériques - Bouton-poussoirs
		\item[Etape 3 - 45 min] Mettre en \oe{}uvre des interruptions sur des événements externes
		\item[Etape 4 - 45 min] Utiliser des sorties modulées en largeur d'impulsion (PWM) - LEDs
		\item[Etape 5 - 60 min] Acquérir des données analogiques - Potentiomètre
	\end{description}	

\subsection{Séance 2 / Prise en main de la maquette et automatisation de la mesure}

	\begin{description}
		\item[Etape 6 - 60 min] Piloter l'intensité des LEDs de la maquette
		\item[Etape 7 - 60 min] Piloter le servomoteur de la maquette
		\item[Etape 8 - 90 min] Définir et tester une première structure de code permettant de piloter le servomoteur en faisant une acquisition de la luminosité à pas régulier
		\item[Etape 9 - 60 min] Récupérer les données à l'aide du moniteur Série du logiciel Arduino
	\end{description}
	
\subsection{Séance 3 / Mise en place d'un protocole de communication}

	\begin{description}
		\item[Etape 10 - 90 min] Mettre en place un protocole de communication basé sur une liste de commandes et intégrer à la structure du code embarqué
		\item[Etape 11 - 90 min] Utiliser la bibliothèque pySerial en Python pour envoyer les commandes à la carte Arduino
		\item[Etape 12 - 90 min] Tester la communication entre la carte Arduino et le script Python pour afficher les données
	\end{description}

\medskip


\subsection{Séance 4 / Application complète}

Cette dernière séance sera consacrée à la finalisation des différents programmes : embarqué sur la carte Arduino pour la mesure automatique et sur l'ordinateur pour le pilotage de la maquette et l'affichage des données.

Selon le temps restant, il sera possible d'intégrer les fonctions écrites en Python dans une interface graphique, dont la structure de base est fournie.


\cleardoublepage
\strut % empty page
%%%%%%%%%%%%%%%%%%%%%%%%%%%%%%%%%%%%%%%%%%%%%%%%
%%%%%%%%%%%%%    Séance 2 détaillée

\begin{minipage}[c]{.25\linewidth}
	\includegraphics[width=4cm]{images/Logo-LEnsE.png}
\end{minipage} \hfill
\begin{minipage}[c]{.4\linewidth}

\begin{center}
\vspace{0.3cm}
{\Large \textsc{Interfaçage Numérique}}

\medskip

6N-047-SCI \qquad \textbf{\large Bloc Rayonnement}

\end{center}
\end{minipage}\hfill

\vspace{0.5cm}

\noindent \rule{\linewidth}{1pt}

{\noindent\Large \rule[-7pt]{0pt}{30pt} Séance 2 / Prise en main de la maquette et automatisation de la mesure} 

\noindent \rule{\linewidth}{1pt}


%%%%%%%%%%%%%%%%%%%%%%%%%%%%%%%%%%%%%%%%%%%%%%%%
%%%%%%%%%%%%%    Objectifs
\section{Objectifs de la séance}

Cette seconde séance est consacrée à la \textbf{prise en main de la maquette} et à l'automatisation de la mesure de l'intensité lumineuse d'une source à LED (par exemple).


%%%%%%%%%%%%%%%%%%%%%%%%%%%%%%%%%%%%%%%%%%%%%%%%
%%%%%%%%%%%%%    Maquette
\section{Description de la maquette}


\subsection{Eléments constitutifs}



\subsection{Alimentation électrique}


\noindent \rule{\linewidth}{1pt}

\bigskip

{\LARGE La tension maximale admissible par le servomoteur est de $6\operatorname{V}$ !}

{\Large Le courant maximal admissible par la LED de puissance est de $200\operatorname{mA}$ !}

\bigskip

\noindent \rule{\linewidth}{1pt}


\subsection{Brochage}

\begin{center}
\begin{tabular}{|l|l|l|l|}
\hline 
Maquette & \textbf{Broche Nucléo} & Type & Description \\ 
\hline 
\textsc{LED1} & PC7 & Sortie / PWM & Led active à l'\textbf{état haut}\\ 
\textsc{LED2} & PB13 & Sortie / PWM & Led active à l'\textbf{état bas}\\ 
\hline 
\textsc{SW1} & PC6 & Entrée & Bouton-poussoir, par défaut état bas\\ 
\textsc{SW2} & PC8 & Entrée & Bouton-poussoir, par défaut état bas\\ 
\textsc{UserButton} & PC13 & Entrée & Bouton-poussoir, par défaut état haut\\
\hline 
\textsc{Servo} & PB7 & Sortie / PWM & Servomoteur - T = 20 ms\\ 
\hline 
\textsc{Photodiode} & PC3 & Entrée analogique & Photodiode \\
 & & & (montage simple avec R variable)\\ 
\hline 
\textsc{Résistance Thermique} & PC1 & Entrée analogique & Résistance Thermique CT10k\\ 
\hline 
\textsc{SPI} & & & \\ 
\textit{SCK} & PA5 & Sortie & Signal d'horloge\\
\textit{MISO} & PA6 & Entrée & Données entrantes\\
\textit{MOSI} & PA7 & Sortie & Données sortantes\\
\hline 
\textsc{Pot Num} & SPI & MCP4132 & Potentiomètre Numérique\\ 
 & & & Monté dans un pont diviseur\\ 
\textit{CS} & PB9 & Sortie & \\ 
\textit{Pot Num In} & PB0 & Entrée analogique & Sortie du pont diviseur\\ 
\hline 
\textsc{LED Puissance} & SPI & MCP4132 & Potentiomètre numérique / Contrôle\\ 
 & & & courant dans Driver BCR430-UW6\\ 
\textit{CS} & PB5 & Sortie & \\
\hline 
\end{tabular} 
\end{center}



\subsubsection{Utilisation de la sortie modulée PB7}
	
\begin{lstlisting}
LL_GPIO_SetAFPin_0_7(GPIOB,  GPIO_PIN_7,  GPIO_AF1_TIM2);
\end{lstlisting}



%%%%%%%%%%%%%%%%%%%%%%%%%%%%%%%%%%%%%%%%%%%%%%%%
%%%%%%%%%%%%%    Etape 6
\section{Etape 6 / Piloter l'intensité des LEDs de la maquette}


%%%%%%%%%%%%%%%%%%%%%%%%%%%%%%%%%%%%%%%%%%%%%%%%
%%%%%%%%%%%%%    Etape 7
\section{Etape 7 / Piloter le servomoteur de la maquette}


%%%%%%%%%%%%%%%%%%%%%%%%%%%%%%%%%%%%%%%%%%%%%%%%
%%%%%%%%%%%%%    Etape 8
\section{Etape 8 / Définir et tester une première structure de code}


%%%%%%%%%%%%%%%%%%%%%%%%%%%%%%%%%%%%%%%%%%%%%%%%
%%%%%%%%%%%%%    Etape 9
\section{Etape 9 / Récupérer les données à l'aide du moniteur Série du logiciel Arduino}





\cleardoublepage
\strut % empty page
%%%%%%%%%%%%%%%%%%%%%%%%%%%%%%%%%%%%%%%%%%%%%%%%
%%%%%%%%%%%%%    Séance 3 détaillée

\begin{minipage}[c]{.25\linewidth}
	\includegraphics[width=4cm]{images/Logo-LEnsE.png}
\end{minipage} \hfill
\begin{minipage}[c]{.4\linewidth}

\begin{center}
\vspace{0.3cm}
{\Large \textsc{Interfaçage Numérique}}

\medskip

6N-047-SCI \qquad \textbf{\large Bloc Rayonnement}

\end{center}
\end{minipage}\hfill

\vspace{0.5cm}

\noindent \rule{\linewidth}{1pt}

{\noindent\Large \rule[-7pt]{0pt}{30pt} Séance 3 / Mise en place d'un protocole de communication} 

\noindent \rule{\linewidth}{1pt}


%%%%%%%%%%%%%%%%%%%%%%%%%%%%%%%%%%%%%%%%%%%%%%%%
%%%%%%%%%%%%%    Objectifs
\section{Objectifs de la séance}

Cette séance a pour but de mettre en place un \textbf{protocole de communication} entre la carte embarquée et l'ordinateur, d'abord à l'aide d'un moniteur série puis à l'aide d'un script en Python.



L'ordinateur, maitre de la communication, devra envoyer des commandes à la carte embarquée, qui devra les exécuter puis acquitter du bon déroulement de l'opération.\

\textit{Un exemple de code pour la transmission des commandes A et D, à analyser et à tester, est fourni.}


%%%%%%%%%%%%%%%%%%%%%%%%%%%%%%%%%%%%%%%%%%%%%%%%
%%%%%%%%%%%%%    Etape 10
\section{Etape 10 / Mettre en place un protocole de communication basé sur une liste de commandes}

Code de base avec les commandes A et D

%%%%%%%%%%%%%%%%%%%%%%%%%%%%%%%%%%%%%%%%%%%%%%%%
%%%%%%%%%%%%%    Etape 11
\section{Etape 11 / Utiliser la bibliothèque pySerial en Python pour envoyer les commandes à la carte Arduino}

Code de base pySerial pour envoyer les commandes A et D et récupérer des données.

%%%%%%%%%%%%%%%%%%%%%%%%%%%%%%%%%%%%%%%%%%%%%%%%
%%%%%%%%%%%%%    Etape 12
\section{Etape 12 / Tester la communication entre la carte Arduino et le script Python pour afficher les données}

A faire : mettre en place la structure de base.




\subsection{Liste des commandes - côté maitre}

\begin{center}

\begin{tabular}{|l|l|l|}
\hline 
\textbf{Commande PC} & Description & \textbf{Réponse Nucléo} \\ 
\hline 
\textsc{!T?} & Test de la transmission & \textsc{!T;} \\ 
\hline 
\textsc{!A:value?} & Transmission de l'angle de départ souhaité pour le diagramme & \textsc{!A:value;} \\ 
 &  valeur entière en degré &  \\
 & \textit{Si angle non compris entre -90 et 90} & \textsc{value}$ = -100$ \\ 
\hline 
\textsc{!B:value?} & Transmission de l'angle final souhaité pour le diagramme & \textsc{!B:value;} \\ 
 &  valeur entière en degré &  \\ 
 & \textit{Si angle non compris entre -90 et 90} & \textsc{value}$ = -100$ \\
\hline 
\textsc{!C:value?} & Transmission du pas angulaire souhaité pour le diagramme & \textsc{!C:value;} \\ 
 &  valeur entière en degré &  \\ 
 & \textit{Si angle non compris entre -90 et 90} & \textsc{value}$ = -100$ \\
\hline 
\textsc{!S?} & Lancement de l'acquisition & \textsc{!S:nb;} \\ 
 &  nb est le nombre d'échantillons qui seront acquis par la carte &  \\ 
 & \textit{Si les angles fournis sont non compatibles} & \textsc{nb}$ = 0$ \\
\hline 
\textsc{!E?} & Test de fin de l'acquisition & \textsc{!E:Y/N;} \\
 &  (Y)es or (N)o &  \\  
\hline 
\textsc{!D:index?} & Demande de récupération d'une donnée & \textsc{!D:index:value;} \\
 &  index correspond au numéro de l'échantillon souhaité &  \\  
 &  value correspond à la valeur de l'échantillon souhaité &  \\  
\hline 
\end{tabular} 

\end{center}


%%%%%%%%%%%%%%%%%%%%%%%%%%%%%%%%%%%%%%%%%%%%%%%%
%%% RESSOURCES COMPLEMENTAIRES		

\newpage
% Ressources
\begin{center}
	\begin{minipage}{2.5cm}
	\begin{center}
		\includegraphics[width=5cm]{images/Logo-LEnsE.png}
	\end{center}
\end{minipage}\hfill
\begin{minipage}{10cm}
	\begin{center}
	\textbf{Institut d'Optique Graduate School }\\[0.1cm]
    \textbf{Interfaçage Numérique}


	\end{center}
\end{minipage}\hfill


\vspace{2cm}


{\Large \bfseries \textsc{Interfaçage Numérique}} \\[0.5cm]
{\large \bfseries Travaux Pratiques} \\[0.2cm]
Semestre 6

\vspace{1cm}

% Title
\rule{\linewidth}{0.4mm} \\[0.4cm]
{ \Large \bfseries\color{violet_iogs} Ressources \\[0.4cm] }
\rule{\linewidth}{0.4mm} \\[1cm]
{\large Bloc Rayonnement}

\end{center}

\vspace{3cm}

\textbf{\large Liste des ressources}
\begin{itemize}
	\item \hyperref[doc:robot_schematic]{Schéma de la carte de la maquette de rayonnement}
	\item \hyperref[doc:robot_pcb]{PCB de la carte de la maquette de rayonnement}
\end{itemize}

\vfill

\newpage
\strut % empty page
% Ressources
\titleformat{\section}
  {\null}{}{0pt}{}


\includepdf[pages=1, landscape=true, pagecommand={\section{\texorpdfstring{\hspace{-1em}}{Schéma Carte Rayonnement}}}\label{doc:robot_schematic}]{ressources/Rayonnement_L476RG_sch.pdf}


\includepdf[pages=1, pagecommand={\section{\texorpdfstring{\hspace{-1em}}{PCB Carte Rayonnement}}}\label{doc:robot_pcb}]{ressources/Rayonnement_L476RG_pcb.pdf}

\end{document}


