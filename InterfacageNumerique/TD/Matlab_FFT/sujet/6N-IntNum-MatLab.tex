%%%%%%%%%%%%%%%%%%%%%%%%%%%%%%%%%%%%%%%%%%
% Engineering problems / LaTeX Template
%		Semester 5
%		Institut d'Optique Graduate School
%%%%%%%%%%%%%%%%%%%%%%%%%%%%%%%%%%%%%%%%%%
%	5N-ONIP-Block1	/ Python for Science
%%%%%%%%%%%%%%%%%%%%%%%%%%%%%%%%%%%%%%%%%%
%
% Created by:
%	Julien VILLEMEJANE - 25/sep/2024
% Modified by:
%	
%
%%%%%%%%%%%%%%%%%%%%%%%%%%%%%%%%%%%%%%%%%%
% Professional Newsletter Template
% LaTeX Template
% Version 1.0 (09/03/14)
%
% Created by:
% Bob Kerstetter (https://www.tug.org/texshowcase/) and extensively modified by:
% Vel (vel@latextemplates.com)
% 
% This template has been downloaded from:
% http://www.LaTeXTemplates.com
%
% License:
% CC BY-NC-SA 3.0 (http://creativecommons.org/licenses/by-nc-sa/3.0/)
%
%%%%%%%%%%%%%%%%%%%%%%%%%%%%%%%%%%%%%%%%%

\documentclass[10pt]{article} % The default font size is 10pt; 11pt and 12pt are alternatives

\input{6N_ONIP_structure.tex} % Include the document which specifies all packages and structural customizations for this template
\usepackage{amsmath}

%----------------------------------------------------------------------------------------
%	DOCUMENT INFORMATIONS
%----------------------------------------------------------------------------------------
\def\module{Interfaçage Numérique}
\def\submodule{IntNum}
\def\moduleSmall{6N-047-SCI / IN}
\def\year{2024-2025}
\def\problem{TD Signaux, images et FFT}
\def\problemName{IntNum / TD Signaux, images et FFT}

\def\validation{}

\def\scheduleCM{0}
\def\scheduleTD{0}
\def\scheduleTDcomputer{1}
\def\scheduleTP{0}

\def\workingTeam{Par binôme}

\def\workingSpecial{}

\def\keywords{Systèmes asservis;ALI;FFT}


\begin{document}
%----------------------------------------------------------------------------------------
%	HEADER IMAGE
%----------------------------------------------------------------------------------------

\begin{figure}[h!]
\centering\includegraphics[width=0.3\linewidth]{logo_iogs.png}
\end{figure}

%----------------------------------------------------------------------------------------
%	MAIN BODY - FIRST PAGE
%----------------------------------------------------------------------------------------
%
\begin{minipage}[t]{1.0\linewidth} % Mini page taking up 65% of the actual page

\hypertarget{context}{\heading{\huge \problemName}{6pt}} % \hypertarget provides a label to reference using \hyperlink{label}{link text}

%-----------------------------------
\centerline {\rule{.70\linewidth}{.25pt}} % Horizontal line

\medskip

Pour ce TD, nous utiliserons l'environnement \textsc{\textbf{MatLab}} de \textbf{Mathworks}. Il utilise un langage de script destiné au \textbf{calcul scientifique}, au même titre que le langage \textit{Python} associé à des bibliothèques de type \textit{Numpy} et \textit{Matplotlib}.

Des fichiers contenant des exemples sont disponibles sur le site du LEnsE dans la rubrique \textit{Année / Première Année / Interfaçage Numérique S6 / TD Interfaçage Numérique / TD Signaux, images et FFT / Codes Matlab}.

\centerline {\rule{.70\linewidth}{.25pt}} % Horizontal line


%-----------------------------------
\textbf{Exercice 0 / Interface de MatLab}

\begin{enumerate}
	\item Lancer l'application MatLab 
	\item Dans la fenêtre \textsl{Command Window} (au centre de l'interface), saisir l'instruction suivante : 

\begin{lstlisting}
v = linspace(0,1,101)
\end{lstlisting}

	\item Que réalise cette instruction ? A quoi correspond la zone \textsl{Workspace} (à droite de l'interface) ?

\end{enumerate}


\centerline {\rule{.70\linewidth}{.25pt}} % Horizontal line
%-----------------------------------
\textbf{Exercice 1 / FFT sur des signaux 1D}

On se propose de tester le code \textsl{exercice1.m}. 

\begin{enumerate}
	\item Créer un script par la commande \textsl{New Script} (en haut à gauche de l'interface)
	\item Tester le script fourni.
	\item Quelle est la fréquence du signal \textsl{ySin1} ? Quelle est la période d'échantillonnage du signal ? Est-ce suffisant ?
	\item A quoi correspond le second graphique ?

	\medskip	
	
	\item Générer un second signal sinusoïdal \textsl{ySin2} de fréquence $287\operatorname{Hz}$ et d'amplitude $3$
	
	\item Tracer ces deux signaux sur un même graphique.

	On se propose d'étudier un signal \textsl{ySinSom} correspondant à la somme de ces deux signaux : $ySinSom = ySin1 + ySin2$
	
	\item Tracer le signal $ySinSom$ sur un graphique.
	
	\medskip
	
	\item Calculer la FFT de ce signal et tracer cette réponse en fréquence sur un nouveau graphique. Construire l'axe des fréquences.
\end{enumerate}



\centerline {\rule{.70\linewidth}{.25pt}} % Horizontal line
%-----------------------------------
\textbf{Exercice 2 / FFT sur des images}

On se propose d'étudier le script \textsl{exercice2.m} (associé à l'image \textsl{test\_image.png}).

\begin{enumerate}
	\item Tester ce script.
	\item A quoi servent les différentes étapes de ce script ? Sont-elles très différentes de celles utilisées en \textit{Python} ?
	
	\medskip
	
	\item Appliquer la fonction \textsl{circular\_mask.py} à la transformée de Fourier de l'image, avec un rayon de 30 pixels.
	\item Générer l'image associée à cette nouvelle transformée de Fourier (\textsl{ifft2()} - voir l'aide de MatLab) et l'afficher.
	\item Tester avec plusieurs tailles de rayon. Inverser également le masque. Expliquer l'impact du choix du masque sur l'image finale.
\end{enumerate}

\end{minipage}

\end{document} 