%%%%%%%%%%%%%%%%%%%%%%%%%%%%%%%%%%%%%%%%%%
% Engineering problems / LaTeX Template
%		Semester 5
%		Institut d'Optique Graduate School
%%%%%%%%%%%%%%%%%%%%%%%%%%%%%%%%%%%%%%%%%%
%	5N-ONIP-Block1	/ Python for Science
%%%%%%%%%%%%%%%%%%%%%%%%%%%%%%%%%%%%%%%%%%
%
% Created by:
%	Julien VILLEMEJANE - 25/sep/2024
% Modified by:
%	
%
%%%%%%%%%%%%%%%%%%%%%%%%%%%%%%%%%%%%%%%%%%
% Professional Newsletter Template
% LaTeX Template
% Version 1.0 (09/03/14)
%
% Created by:
% Bob Kerstetter (https://www.tug.org/texshowcase/) and extensively modified by:
% Vel (vel@latextemplates.com)
% 
% This template has been downloaded from:
% http://www.LaTeXTemplates.com
%
% License:
% CC BY-NC-SA 3.0 (http://creativecommons.org/licenses/by-nc-sa/3.0/)
%
%%%%%%%%%%%%%%%%%%%%%%%%%%%%%%%%%%%%%%%%%

\documentclass[10pt]{article} % The default font size is 10pt; 11pt and 12pt are alternatives

\input{6N_ONIP_structure.tex} % Include the document which specifies all packages and structural customizations for this template
\usepackage{amsmath}

%----------------------------------------------------------------------------------------
%	DOCUMENT INFORMATIONS
%----------------------------------------------------------------------------------------
\def\module{Interfaçage Numérique}
\def\submodule{IntNum}
\def\moduleSmall{6N-047-SCI / IN}
\def\year{2024-2025}
\def\problem{TD Conversion Analogique Numérique}
\def\problemName{IntNum / TD Conversion Analogique Numérique}

\def\validation{}

\def\scheduleCM{0}
\def\scheduleTD{1}
\def\scheduleTDcomputer{0}
\def\scheduleTP{0}

\def\workingTeam{}

\def\workingSpecial{}

\def\keywords{Conversion analogique-numérique; Echantillonnage; Transmission}


\begin{document}
%----------------------------------------------------------------------------------------
%	HEADER IMAGE
%----------------------------------------------------------------------------------------

\begin{figure}[h!]
\centering\includegraphics[width=0.3\linewidth]{logo_iogs.png}
\end{figure}

%----------------------------------------------------------------------------------------
%	MAIN BODY - FIRST PAGE
%----------------------------------------------------------------------------------------
%

\hypertarget{context}{\heading{\huge \problemName}{6pt}} % \hypertarget provides a label to reference using \hyperlink{label}{link text}

%-----------------------------------
\centerline {\rule{.70\linewidth}{.25pt}} % Horizontal line
%-----------------------------------
\textbf{Exercice 1 / Données numériques}

Les images couleurs sont composées de \textbf{pixels}, chacun codé en \textbf{R}ouge, \textbf{V}ert et \textbf{B}leu. Chacune des couleurs est codée sur \textbf{8 bits}. Les formats des images utilisées dans le domaine de la vidéo numérique sont les suivants (plateforme de \textit{streaming} par exemple) :

\begin{center}
\begin{tabular}{|c|c|c|c|}
\hline
\textbf{ 480p}  720 x 480 pixels & 
\textbf{ 720p}  1280 x 720 pixels & 
\textbf{Full HD}  1920 x 1080 pixels & 
\textbf{4K}  3840 x 2160 pixels \\
\hline
\end{tabular}
\end{center}

Ces images sont rafraichies à un rythme de \textbf{25 images/seconde}.

\medskip

\begin{enumerate}
	\item Sur combien d'octets sont codés chacun des pixels ?
	\item Quelle taille, en octets, faut-il pour stocker une image en 4K sur un support physique ? Une image en 720p ?
	\item Quelle taille, en octets, faut-il pour stocker une seconde de vidéo en 4K sur un support physique ? Une seconde de vidéo en 720p ?
	
	\bigskip
	
	Les débits en réception des différents moyens de communication actuels sont les suivants (valeur moyenne - décembre 2024) :

\begin{center}
\begin{tabular}{|c|c|}
\hline
\textbf{ Fibre Optique}  573 Mbits/s & 
\textbf{ Réseau 5G}  500 Mbit/s \\
\hline
\end{tabular}
\end{center}

	\medskip
	
	\item Dans votre colocation, vous êtes 2 et vous souhaitez regarder deux vidéos différentes. Quelle qualité vidéo pouvez-vous utiliser à l'aide de votre connexion par fibre optique ?
	\item Une coupure de votre routeur vous oblige à passer chacun sur votre téléphone 5G. Quelle est la qualité vidéo maximale utilisable ?
\end{enumerate}

\textit{On supposera dans cet exercice que les images sont \textbf{non compressées}. Il existe cependant des encodages permettant des réduction de 40\% sans perte en moyenne (\textbf{FFV1}) à 90\% avec perte (\textbf{H.264}).}

\centerline {\rule{.70\linewidth}{.25pt}} % Horizontal line
%-----------------------------------
\textbf{Exercice 2 / Conversion analogique-numérique}

Soit le signal suivant. 

\begin{center}
	\includegraphics[width=0.6\textwidth]{images/can_signal.png}
\end{center}

On souhaite l'encoder sur sur 8 niveaux entre les valeurs \textsl{MIN} et \textsl{MAX}. Les échantillons [i] sont pris à intervalle régulier

\begin{enumerate}
	\item Combien de bits faut-il pour transmettre un échantillon ?
	\item Graduer l'axe des ordonnées avec les valeurs obtenues en sortie du convertisseur analogique-numérique (valeurs binaire et décimale).
	\item Quelles sont les valeurs binaires et décimales des 8 premiers échantillons ?
\end{enumerate}



\centerline {\rule{.70\linewidth}{.25pt}} % Horizontal line
%-----------------------------------
\textbf{Exercice 3 / Transmission numérique}

On souhaite transmettre des informations binaires sur une fibre. Le laser d'émission peut être piloté selon 4 niveaux d'intensité lumineuse. On ajoute également la possibilité de choisir 2 états de polarisation. 

\textit{On supposera que le délai de changement de niveaux de luminosité et de polarisation n'est pas un facteur limitant de la transmission.}

\begin{enumerate}
	\item Quelle est la valence de ce mode de transmission ?
	\item Quelle est la quantité de bits transmis par motif ?
	\item Chaque motif reste un temps $\Delta_T$ sur la fibre. En déduire le débit binaire en bits/s puis en octets/s. 
	
	\medskip	
	
	AN : $\Delta_T = 100\operatorname{ns}$
\end{enumerate}

\centerline {\rule{.70\linewidth}{.25pt}} % Horizontal line
%-----------------------------------
\textbf{Exercice 4 / Conversion numérique-analogique}

\textbf{Montage R-2R}

On s'intéresse à ce montage :

\begin{center}
	\includegraphics[width=0.5\textwidth]{images/R_2R.png}
\end{center}

\begin{enumerate}
	\item Que vaut le courant $I_1$ en fonction du courant $I_0$ (courant passant par la résistance $2R$) ?
	\item Que vaut le courant $I_2$ en fonction du courant $I_0$ (courant passant par la résistance $2R$) ?
\end{enumerate}

\textbf{Montage complet}

On s'intéresse à présent au montage suivant :

\begin{center}
	\includegraphics[width=0.8\textwidth]{images/R_2R_complet.png}
\end{center}

On supposera que lorsque $A_i = 0$, l'interrupteur $i$ est en position 3 et que lorsque Ai = 1, l'interrupteur $i$ est en position 1.

\begin{enumerate}
	\item Quel est le type de montage autour de l'ALI ?
	\item En quoi la structure vue précédemment peut nous aider ?
	\item Que vaut alors le courant $I_{tot}$ dans la contre-réaction de l'ALI en fonction des courants $I_i$ ?
	\item Que vaut alors le courant $I_{tot}$ dans la contre-réaction de l'ALI en fonction du courant $I_0$ et des
valeurs des $A_i$ ?
\end{enumerate}

\end{document} 