%%%%%%%%%%%%%%%%%%%%%%%%%%%%%%%%%%%%%%%%%%
% Engineering problems / LaTeX Template
%		Semester 5
%		Institut d'Optique Graduate School
%%%%%%%%%%%%%%%%%%%%%%%%%%%%%%%%%%%%%%%%%%
%	5N-ONIP-Block1	/ Python for Science
%%%%%%%%%%%%%%%%%%%%%%%%%%%%%%%%%%%%%%%%%%
%
% Created by:
%	Julien VILLEMEJANE - 25/sep/2024
% Modified by:
%	
%
%%%%%%%%%%%%%%%%%%%%%%%%%%%%%%%%%%%%%%%%%%
% Professional Newsletter Template
% LaTeX Template
% Version 1.0 (09/03/14)
%
% Created by:
% Bob Kerstetter (https://www.tug.org/texshowcase/) and extensively modified by:
% Vel (vel@latextemplates.com)
% 
% This template has been downloaded from:
% http://www.LaTeXTemplates.com
%
% License:
% CC BY-NC-SA 3.0 (http://creativecommons.org/licenses/by-nc-sa/3.0/)
%
%%%%%%%%%%%%%%%%%%%%%%%%%%%%%%%%%%%%%%%%%

\documentclass[10pt]{article} % The default font size is 10pt; 11pt and 12pt are alternatives

\input{6N_ONIP_structure.tex} % Include the document which specifies all packages and structural customizations for this template
\usepackage{amsmath}

%----------------------------------------------------------------------------------------
%	DOCUMENT INFORMATIONS
%----------------------------------------------------------------------------------------
\def\module{Interfaçage Numérique}
\def\submodule{IntNum}
\def\moduleSmall{6N-047-SCI / IN}
\def\year{2024-2025}
\def\problem{TD Caméra et Images}
\def\problemName{IntNum / TD Caméra et Images}

\def\validation{}

\def\scheduleCM{0}
\def\scheduleTD{1}
\def\scheduleTDcomputer{0}
\def\scheduleTP{0}

\def\workingTeam{}

\def\workingSpecial{}

\def\keywords{Capteur CMOS; Images; Convolution; Erosion}


\begin{document}
%----------------------------------------------------------------------------------------
%	HEADER IMAGE
%----------------------------------------------------------------------------------------

\begin{figure}[h!]
\centering\includegraphics[width=0.3\linewidth]{logo_iogs.png}
\end{figure}

%----------------------------------------------------------------------------------------
%	MAIN BODY - FIRST PAGE
%----------------------------------------------------------------------------------------
%

\hypertarget{context}{\heading{\huge \problemName}{6pt}} % \hypertarget provides a label to reference using \hyperlink{label}{link text}

%-----------------------------------
\centerline {\rule{.70\linewidth}{.25pt}} % Horizontal line
%-----------------------------------
\textbf{Exercice 1 / Caméra CMOS}

\begin{enumerate}
	\item Faire un schéma représentant les éléments constitutifs d'un pixel d'une caméra CMOS.
	
	\bigskip
	
	On souhaite, à l'aide d'une caméra CMOS, visualiser un objet pouvant être contenu dans un carré de $4.5\operatorname{cm}$ de côté.
	
	On veut pouvoir mesurer des distances à une précision de l'ordre du $\operatorname{mm}$. On prendra une résolution de $5\operatorname{pixels/mm}$.
	
	\item Quelle doit être la résolution minimale du capteur ?
	
	On choisit une caméra dont les pixels font $3\operatorname{\mu{}m}$ de côté.
	
	\item Quelle est la taille du capteur ?
	
	L'objet sera placé à $20\operatorname{cm}$ du capteur. On souhaite à présent caractériser l'objectif à placer devant le capteur.
	
	\item Faire un schéma de principe de l'objet et de l'image de l'objet (optique instrumentale).	
	\item Quel grandissement faut-il pour répondre au cahier des charges ?
	\item Quelle focale faut-il choisir pour l'objectif ?
	
\end{enumerate}

\centerline {\rule{.70\linewidth}{.25pt}} % Horizontal line
%-----------------------------------
\textbf{Exercice 2 / Opérations morphologiques - Erosion et Dilatation}

L'\textbf{érosion} et la \textbf{dilatation} sont deux opérations de base en morphologie mathématique, utilisées pour le pré-traitement et l'analyse d'images.

\medskip

L'\textbf{érosion} est une opération qui réduit les objets présents dans l'image. Elle consiste à appliquer un \textbf{élément structurant} (ou noyau) à chaque pixel et à ne conserver un pixel que si tous les pixels couverts par l'élément structurant correspondent à l'objet (opération booléenne ET).

La \textbf{dilatation} est l'opération inverse de l'érosion. Elle élargit les objets dans l'image. Un pixel devient un pixel de l'objet s'il y a au moins un pixel de l'objet sous l'élément structurant (opération booléenne OU).


\begin{enumerate}
	\item A partir des noyaux proposés dans les pages suivantes, réaliser l'opération morphologique d'érosion et de dilatation sur l'exemple donné.
	\item Quel noyau utilisé pour détecter des lignes verticales ? Tester sur les exemples fournis en annexe.	
\end{enumerate}

\includepdf[pages={1,2}, pagecommand={\section{\texorpdfstring{\hspace{-1em}}{Image Processing}}}\label{doc:image_proc}]{../Feuille_exo_TD.pdf}


\end{document} 