\documentclass[a4paper]{book}%
\input{6N_IntNum_structure.tex}
\usepackage{amsmath}

%----------------------------------------------------------------------------------------
%	DOCUMENT
%----------------------------------------------------------------------------------------
\begin{document}

%----------------------------------------------------------------------------------------
% SUJET 1
%----------------------------------------------------------------------------------------

	\Large 
\begin{tabular}{c}
\includegraphics[width=5cm]{logoLEnsE.png}\\
Cycle ingénieur 1A
\end{tabular}
\hfill
\begin{tabular}{c}
Travaux Pratiques \\
\textbf{Interfaçage Numérique} \\
Semestre 6 \\
\end{tabular}\\
\normalsize 

\bigskip

\begin{mdframed}[style=aavbox,frametitle={Test individuel}]
	
L'objectif principal de ce test est l'\textbf{auto-évaluation} de l'acquisition individuelle des savoirs et savoir-faire dans le domaine des systèmes embarqués.

Vous avez 2 heures pour traiter ce sujet en \textbf{autonomie} :
\begin{itemize}
	\item \textbf{concevoir et réaliser un circuit mixte} (analogique et numérique) sur une plaquette de prototypage, incluant une carte \textbf{\textit{Nucléo L476RG}}
	\item proposer un \textbf{protocole expérimental de validation}
	\item mettre en \oe{}uvre un \textbf{protocole de mesure adapté}
	\item \textbf{analyser} les mesures réalisées
\end{itemize}

Une grille d'auto-évaluation est fournie au verso de cette page. 

\textbf{Vous avez accès à toutes les ressources documentaires.}
\end{mdframed}	

\medskip	




	\noindent \hrulefill
	
	\begin{large}

\textbf{ATTENTION}

Les tensions admissibles par les entrées de la carte Nucléo doivent être comprises entre 0 et 3.3V.
	
	\end{large}

	\noindent \hrulefill
	
	
	\begin{large}

On souhaite tester le fonctionnement du code fourni dans le fichier 
\textsl{\textbf{test\_nucleo.cpp}}. Ce fichier est disponible sur le site du LEnsE dans la rubrique \textit{Année / Première Année / Interfaçage Numérique S6 / Bloc 1 Systèmes embarqués / Auto-Evaluation / Code de test}.
	
	\begin{enumerate}
		\item Créer un nouveau projet sur \textbf{Keil Studio Cloud} (ou MBed Studio).
		\item Remplacer le contenu du fichier \textsl{main.cpp} du projet par le contenu du fichier \textsl{test\_nucleo.cpp}.
		\item Proposer un protocole de test de cette application embarquée. \textit{Vous pouvez vous inspirer de vos précédentes réalisations et utiliser des composants annexes (LED, bouton-poussoir, oscilloscope...).}
		
		\medskip
		
		\item Mettre en \oe{}uvre ce protocole.
		\item Expliquer le fonctionnement de ce programme, en justifiant notamment le rôle du \textsf{Ticker} et de la sortie \textsl{outS2}.
		
	\bigskip
	
	
On souhaite à présent ajouter la fonctionnalité suivante au programme.

L'appui sur le bouton-poussoir bleu de la carte (entrée numérique \textsl{PC\_13}, nommée \textsl{inBP} dans le programme) permet de valider l'utilisation de la sortie \textsl{outS1} (LED LD2 de la carte). Lors d'un premier appui, le fonctionnement précédent
est obtenu. Lors d'un second appui, on force la sortie \textsl{outS1} à 0.
	
	\medskip

		\item Mettre en \oe{}uvre cette fonctionnalité. \textit{La boucle while(true) ne doit pas être modifiée...}
	
	\bigskip
	
	Question complémentaire :
	
	\medskip
	
		\item Proposer et mettre en \oe{}uvre un protocole de test permettant de mesurer le temps d'exécution de l'instruction de conversion analogique-numérique (\textls{read\_u16()}).
	
	\end{enumerate}

	\end{large}
	
	
\includepdf[pages=-,landscape=true]{../S6_IntNum_2025_AutoEval_Indiv.pdf}	



\end{document}