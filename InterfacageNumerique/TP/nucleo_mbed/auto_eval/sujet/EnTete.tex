\Large 
\begin{tabular}{c}
\includegraphics[width=5cm]{logoLEnsE.png}\\
Cycle ingénieur 1A
\end{tabular}
\hfill
\begin{tabular}{c}
Travaux Pratiques \\
\textbf{Interfaçage Numérique} \\
Semestre 6 \\
\end{tabular}\\
\normalsize 

\bigskip

\begin{mdframed}[style=aavbox,frametitle={Test individuel}]
	
L'objectif principal de ce test est l'\textbf{auto-évaluation} de l'acquisition individuelle des savoirs et savoir-faire dans le domaine des systèmes embarqués.

Vous avez 2 heures pour traiter ce sujet en \textbf{autonomie} :
\begin{itemize}
	\item \textbf{concevoir et réaliser un circuit mixte} (analogique et numérique) sur une plaquette de prototypage, incluant une carte \textbf{\textit{Nucléo L476RG}}
	\item proposer un \textbf{protocole expérimental de validation}
	\item mettre en \oe{}uvre un \textbf{protocole de mesure adapté}
	\item \textbf{analyser} les mesures réalisées
\end{itemize}

Une grille d'auto-évaluation est fournie au verso de cette page. 

\textbf{Vous avez accès à toutes les ressources documentaires.}
\end{mdframed}	

\medskip