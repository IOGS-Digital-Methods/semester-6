%%%%%%%%%%%%%%%%%%%%%%%%%%%%%%%%%%%%%%%%%%
% Engineering problems / LaTeX Template
%		Semester 6
%		Institut d'Optique Graduate School
%%%%%%%%%%%%%%%%%%%%%%%%%%%%%%%%%%%%%%%%%%
%	6N-IntNum-BlocVI
%%%%%%%%%%%%%%%%%%%%%%%%%%%%%%%%%%%%%%%%%%
%
% Created by:
%	Julien VILLEMEJANE - 20/nov/2025
%	
%
%%%%%%%%%%%%%%%%%%%%%%%%%%%%%%%%%%%%%%%%%%
% Professional Newsletter Template
% LaTeX Template
% Version 1.0 (09/03/14)
%
% Created by:
% Bob Kerstetter (https://www.tug.org/texshowcase/) and extensively modified by:
% Vel (vel@latextemplates.com)
% 
% This template has been downloaded from:
% http://www.LaTeXTemplates.com
%
% License:
% CC BY-NC-SA 3.0 (http://creativecommons.org/licenses/by-nc-sa/3.0/)
%
%%%%%%%%%%%%%%%%%%%%%%%%%%%%%%%%%%%%%%%%%

\documentclass[a4paper,11pt,titlepage]{article} % The default font size is 10pt; 11pt and 12pt are alternatives

%%%%%%%%%%%%%%%%%%%%%%%%%%%%%%%%%%%%%%%%%%%%%%%%%%%%%%%%%%%%%%%%%%%%%%%%%%%%%%%%%%%%%%%%%%%%%%%%%%%%%%%%%%%%%%%%%%%%%%%%%%%%%%%%%%%%%%%%%%%%%%%%%%%%%%%%%%%%%%%%%%%%%%%%%%%%%%%%%%%%%%%%%%%%%%%%%%%%%%%%%%%%%%%%%%%%%%%%%%%%%%%%%%%%%%%%%%%%%%%%%%%%%%%%%%%%
\usepackage{../opto_elec_villemejane}

%%%%%%%%%%%%%%%%%%%%%%%%%%%%%%%%%%%%%%%%%%%%%%%%
%%%%%%%%%%%%%%%%%%%%%%%%%%%%%%%%%%%%%%%%%%%%%%%%
\begin{document}


% Page de garde
\begin{titlepage}

\begin{center}
	\begin{minipage}{2.5cm}
	\begin{center}
		\includegraphics[width=8cm]{../images/Logo-LEnsE.png}
	\end{center}
\end{minipage}\hfill
\begin{minipage}{10cm}
	\begin{center}
	\textbf{Institut d'Optique Graduate School }\\[0.1cm]
    \textbf{\textsc{\Large Interfaçage Numérique}}


	\end{center}
\end{minipage}\hfill


\vspace{1.5cm}


{\large \bfseries Travaux Pratiques} \\[0.2cm]
Semestre 6

\vspace{1.5cm}
% Title
\rule{\linewidth}{0.3mm} \\[0.4cm]
{ \huge \bfseries\color{violet_iogs} Vision Industrielle \\[0.4cm] }
\rule{\linewidth}{0.3mm} \\[0.2cm]
{ \large \bfseries\color{violet_iogs} Objectifs des 4 séances\\
TP 3-4 / Détection d'objets}
\rule{\linewidth}{0.3mm} \\[1cm]


\bigskip

\begin{center}
	\includegraphics[width=0.9\textwidth]{../images/vi_shapes_intro.png}
\end{center}

\vfill

\textit{Ce sujet est disponible au format électronique sur le site du LEnsE - https://lense.institutoptique.fr/ dans la rubrique Année / Première Année / Interfaçage Numérique S6 / Bloc 2 Vision Industrielle.}

\vspace{1.5cm}

\begin{minipage}{5cm}
\begin{center}
\includegraphics[width=3cm]{../images/logocc}\\
\small
  © 2025 by LEnsE-IOGS 
\end{center}
\end{minipage}

% Bottom of the page
%{\textbf{\large {Année universitaire} 2024-2025}}

\end{center}
\end{titlepage}

\newpage
\strut % empty page


\textit{Ce sujet a été co-conçu par une équipe d'étudiant$\cdot$es - Joséphine BECHU, Justine GABRIEL et Paul CHENEAU - lors d'un projet DEPhI de 2A - 2025-2026 - et l'équipe pédagogique d'Interfaçage Numérique.}



\newpage
\strut % empty page
%%%%%%%%%%%%%%%%%%%%%%%%%%%%%%%%%%%%%%%%%%%%%%%%
%%%%%%%%%%%%%    A A V

\section{Vision Industrielle}

La \textbf{vision industrielle} est une technologie qui permet à des machines d'\textbf{analyser automatiquement des scènes} pour \textbf{contrôler}, \textbf{guider} ou \textbf{inspecter} des objets sur des processus de production. Elle repose sur l'utilisation de \textbf{caméras}, d'\textbf{optique}, d'\textbf{éclairages} spécifiques (ou contraints), de \textbf{capteurs} et d'algorithmes de \textbf{traitement d'image}. 

\begin{center}
	\includegraphics[width=0.5\textwidth]{../images/vi_intro_2.png}
\end{center}

Elle a pour but de \textbf{prendre des décisions automatiques} (ou aider l'être humain dans sa prise de décision) vis-à-vis d'un (ou plusieurs) objet(s) dans une scène spécifique : détecter des défauts ou des irrégularités, compter ou trier..., en rejetant ou validant automatiquement des produits, tout en assurant une constance de la qualité et de la répétabilité des opérations. 


\begin{center}
	\includegraphics[width=0.6\textwidth]{../images/vi_chaine.png}
\end{center}


\medskip

\section{Acquis d'Apprentissages Visés (AAV)}
%%%%%%%%%%%%%%%%%%%%%%%%%%%%%%%%%%%%%%%%%%%%%%%%
%%%%%%%%%%%%%    A A V

À la fin de cette série de 4 séances, les étudiant$\cdot$es seront capable de :

\begin{itemize}
	\item Décomposer et paramétrer une chaîne de vision industrielle complète (du capteur au traitement de l'image),
	\item Comprendre les compromis physiques et numériques de chaque maillon,
	\item Réaliser un prototype d'inspection ou de mesure simple (tri d'objets),
	\item Justifier leurs choix de configuration (résolution, focale, éclairage...)
\end{itemize}

\medskip

%%%%%%%%%%%%%%%%%%%%%%%%%%%%%%%%%%%%%%%%%%%%%%%%
%%%%%%%%%%%%%    Ressources

\newpage
\section{Ressources}

Un \textbf{kit d'images} est disponible sur le site du LEnsE dans la rubrique  \textit{Année / Première Année / Interfaçage Numérique S6 / Bloc 2 Vision Industrielle / Kit d'images}. 

\medskip

Des {codes en Python}, proposant des exemples à tester, sont disponibles sur le site du LEnsE dans la rubrique \textit{Année / Première Année / Interfaçage Numérique S6 / Bloc 2 Vision Industrielle / Répertoire vers codes à tester}. 

\textit{Un fichier archivé, nommé \textsc{\_step\_by\_step.zip}, regroupe l'ensemble des codes à tester au cours de cette séance, ainsi que les images à traiter.}


%%%%%%%%%%%%%%%%%%%%%%%%%%%%%%%%%%%%%%%%%%%%%%%%
%%%%%%%%%%%%%    Déroulement

\section{Déroulement}

Les sujets \textbf{TP1} et \textbf{TP2} se font en binôme et sont interchangeables (4 binômes commenceront par la TP1 lors de la première séance et les 4 autres binômes commenceront par le TP2).

Les sujets \textbf{TP3} et \textbf{TP4} se font par groupe de 4 étudiant$\cdot$es.


\subsection{TP1 - Banc de vision industrielle}

Le \textbf{TP1} se fera sur un banc de vision industrielle simple incluant une caméra, un objectif (focale : 5 / 8 / 10 mm), un éclairage Effilux Ring-RGB et un ordinateur.

\medskip

\textit{Voir le sujet de la séance TP1. Le sujet est disponible au format électronique sur le site du LEnsE - https://lense.institutoptique.fr/ dans la rubrique Année / Première Année / Interfaçage Numérique S6 / Bloc 2 Vision Industrielle.}


\subsection{TP2 - Manipulations de base sous OpenCV}

Le \textbf{TP2} se fera sur un ordinateur possédant une installation de Python (3.11 ou 3.13) et des bibliothèques standard, dont OpenCV.

\medskip

\textit{Voir le sujet de la séance TP2. Le sujet est disponible au format électronique sur le site du LEnsE - https://lense.institutoptique.fr/ dans la rubrique Année / Première Année / Interfaçage Numérique S6 / Bloc 2 Vision Industrielle.}


\subsection{TP3-4 - Détection d'objets}

\textit{Le détail des séances 3 et 4 est donné dans la suite de ce document.}

\section{Accumulation de preuves / Méthode de travail}

Il est conseillé pour ces TP de \textbf{créer un nouveau projet PyCharm} sur votre session (\textit{attention à l'endroit où vous stockerez ce projet - U: sur les sessions Windows de l'IOGS}).

Les différents algorithmes que vous serez amenés à modifier ou créer, pourront vous resservir dans d'autres projets. Nous vous conseillons donc fortement de les \textbf{sauvegarder} précieusement et de les \textbf{commenter} autant que possible afin de retrouver rapidement les principes mis en jeu derrière les fonctionnalités de OpenCV.

\medskip

Il serait également pertinent de votre part de rédiger un \textbf{journal de bord} sur cette série de TP en incluant les résultats (images, histogrammes...) et vos analyses des fonctionnalités et de leur intérêt en traitement d'images.

\newpage
\pagestyle{empty}


%%%%%%%%%%%%%%%%%%%%%%%%%%%%%%%%%%%%%%%%%%%%%%%%
%%%%%%%%%%%%%    A A V / Objectifs

\section{Objectif des séances 3 et 4}

\subsection{TP3 - Manipulations avancées sous OpenCV / Limites du banc de vision}

Sous OpenCV :

\begin{itemize}
	\item Détecter des formes {[40 min]}
	\item Détecter des couleurs {[40 min]}
	\item Changer d'espace de couleur {[40 min]}
\end{itemize}

Segmentation de l'image ?

Sur le banc de VI :

\begin{itemize}
	\item Calibration de la chaîne sur les cubes de couleurs (couleur) {[40 min]}
	\item Calibration de la chaîne sur des formes colorées {[40 min]}
	\item Calibration de la chaîne pour des mesures de distance {[40 min]}
\end{itemize}


\subsection{TP4 - Détection d’objets - Contrôle qualité}

Le but est à partir d'une série d'images (1 en RGB ou 3 en niveau de gris sous éclairage particulier) de détecter le nombre d'objets d'une forme et d'une couleur particulière et de valider si les pièces requises pour un assemblage sont bien présentes (simuler le packaging d'un produit fini - 6 pièces de forme et de couleurs distinctes par exemple).

La validation du packaging final devra se faire sans l’intervention de l'être humain mais un affichage propre des couleurs devra être fait pour le contrôle par le/la manipulateur$\cdot$trice.

Bonus : mesure de la conformité de la taille des pièces ?

\bigskip

Premiers essais sur une grille de jeu TicTacToe (morpion) :

\begin{center}
	\includegraphics[width=0.8\textwidth]{../images/morpion_vi_thres.png}
\end{center}


\newpage
%%%%%%%%%%%%%%%%%%%%%%%%%%%%%%%%%%%%%%%%%%%%%%%%
%%%%%%%%%%%%%    Etape par étape


\section{OpenCV - Détection de formes et de couleurs}

Une partie des algorithmes de traitement sont donnés. Il est fortement conseillé de \textbf{réaliser une analyse poussée} de chacune des étapes de ces algorithmes afin de comprendre l'intérêt de l'utilisation de ces techniques de base.

Vous pouvez par exemple \textbf{décrire ces algorithmes} par l'intermédiaire d'un graphique (algorigramme) et \textbf{afficher les résultats intermédiaires} (et en faire une capture d'écran) tout en \textbf{commentant les résultats}. 

%%%%%%%%%%%%%    Etape A
\subsection{CV-A - Détecter des formes {[40 min]}}

\subsubsection{Détection de contours sur une image "propre"}

\Manip Ouvrir le fichier \textsc{20\_detection\_forme.py} du répertoire des codes à tester.

\Manip Exécuter ce code.

\Quest Que fait ce programme ? A quoi correspondent les images affichées ? A quoi correspondent les données affichées ?

\noindent \rule{\linewidth}{1pt}

\Manip Modifier le paramètre MIN\_AREA (100, 1000, 10000).

\Quest A quoi sert ce paramètre ? Quelle est son unité ? Comment la transformer en unités plus conventionnelles ?

\noindent \rule{\linewidth}{1pt}

\Manip Modifier le paramètre APPROX\_FACTOR (0.01, 0.05, 0.1).

\Quest A quoi sert ce paramètre ?

\Quest A quoi correspondent les variables \texttt{cX}, \texttt{cY}, \texttt{area} et \texttt{peri} ?


\subsubsection{Détection de cercles}

\Manip Ouvrir le fichier \textsc{21\_detection\_forme\_cercles.py} du répertoire des codes à tester.

\Manip Exécuter ce code.

\Quest Quelle amélioration apporte ce programme ?

\subsubsection{Détection de contours sur une image "réelle"}

L'image précédente était "propre" (sans bruits dûs au capteur par exemple). Nous allons nous intéresser à présent à une image provenant d'une chaine d'acquisition.

\Manip A partir du code précédent, ouvrir l'image \textsc{formes\_blanc\_30ms.png} du répertoire du kit d'images.

\Quest L'algorithme est-il efficace ?

\noindent \rule{\linewidth}{1pt}

\Manip Ouvrir le fichier \textsc{22\_detection\_forme\_plus.py} du répertoire des codes à tester.

\Manip Exécuter ce code.

\Quest Quelle étape est-elle ajoutée par rapport à la version précédente du code ?

\noindent \rule{\linewidth}{1pt}

\Manip Ajouter un traitement morphologique de type \textit{opening} (ouverture) sur l'image binarisée (noyau carré de 3x3, puis 5x5).

\Manip Ajouter l'affichage de l'image ainsi traitée.

\Quest Que permet cette étape dans l'algorithme de détection ? 

\Quest L'algorithme de détection de cercles fonctionne-t-il avec cette étape supplémentaire ?


\subsubsection{Détection de cercles sur une image "réelle"}

L'\textbf{algorithme de Hough} qui permet la recherche de cercles dans une image se base sur des images en nuance de gris (et non binarisée). Il s'appuie sur des gradients d'intensité et des bords continus et doux pour détecter les transitions (contours des cercles).

Il n'est donc pas possible d'utiliser une opération morphologique d'ouverture sur l'image. 


\Manip Ouvrir le fichier \textsc{23\_detection\_forme\_blur\_cercles.py} du répertoire des codes à tester.

\Manip Exécuter ce code. Vérifier que l'algorithme n'est pas efficace.

\noindent \rule{\linewidth}{1pt}

\Manip Ajouter un filtrage de type \textit{blur} gaussien (\textit{cv2.GaussianBlur}) sur l'image grise (noyau carré de 7x7 et $\sigma = 1.9$).

\Manip Ajouter l'affichage de l'image ainsi traitée.

\Quest Que permet cette étape dans l'algorithme de détection ?

\Manip Modifier la taille du noyau (3x3, 5x5, 15x15) et de la valeur de $\sigma$ (1.9, 3, 5, 10).

\Quest Quel est l'impact du choix de ces deux paramètres ?


%%%%%%%%%%%%%    Etape B
\subsection{CV-B - Détecter des couleurs {[40 min]}}

\subsubsection{Jouer avec les canaux de couleurs}

\Manip Ouvrir et afficher l'image \textsl{couleurs\_4.png} du kit d'images fourni, au format RGB.

\Manip Créer une copie de la matrice image (fonction \textsl{copy()} de Numpy). 

\Manip Forcer à 0 tous les pixels du canal bleu de la copie de l'image et afficher la nouvelle image et afficher la nouvelle image.



\noindent \rule{\linewidth}{1pt}



%%%%%%%%%%%%%    Etape C
\subsection{CV-C - Changement d'espace de couleur {[40 min]}}

Plusieurs méthodes existent pour passer d'une image RVB (ou RGB) à une image en niveau de gris :

\begin{itemize}
	\item Calculer la \textbf{moyenne des valeurs} des trois canaux de couleur (Rouge, Vert, Bleu) pour chaque pixel.
	\item Utiliser des \textbf{poids spécifiques pour les canaux R, V et B}, basés sur leur contribution relative à la perception humaine.
	\item Convertir l'image dans un \textbf{autre espace de couleur}, comme YUV, HSL ou HSV, et extraire la composante de luminosité.
\end{itemize}

\subsubsection{Moyenne des canaux R,V,B}

Cette méthode est la plus simple. Chaque pixel de l'image en gris est la moyenne des pixels des canaux rouge, vert et bleu de l'image en couleur :

$$Pixel_{Gray} = \frac{Pixel_{R} + Pixel_{V} + Pixel_{B}}{3}$$ 

\Manip Créer une image en nuance de gris utilisant la méthode de la moyenne des trois canaux.

\Manip Afficher l'image résultante.


\subsubsection{Pondération en fonction de la perception humaine}

Cette méthode est une moyenne pondérée des valeurs des pixels R, V, B de l'image couleur : 

$$Pixel_{Gray} = 0.299 \cdot Pixel_{R} + 0.587 \cdot Pixel_{V} + 0.114 \cdot Pixel_{B}$$ 


Les coefficients de cette méthode proviennent de la sensibilité relative de l'œil humain aux différentes couleurs et ont été standardisés à l'origine pour la télévision analogique (NTSC). Ils sont aujourd'hui utilisés comme une approximation fidèle de la perception visuelle de la luminosité.

La méthode de conversion fournie par la bibliothèque OpenCV se base sur cette pondération.

\Manip Convertir l'image RVB en niveau de gris par l'instruction suivante :

\begin{lstlisting}
image_gray = cv2.cvtColor(image_rgb, cv2.COLOR_BGR2GRAY)
\end{lstlisting}

\Manip Comparer les images obtenues par la moyenne classique et cette moyenne pondérée.


\subsubsection{Utilisation d'un espace colorimétrique différent}

L'espace colorimétrique RVB est très utilisé dans le domaine du numérique (affichage, acquisition d'images) pour sa facilité de mise en oeuvre.

Cependant, ce n'est \textbf{pas} le plus \textbf{adapté vis-à-vis de la perception humaine} où la luminance et la couleur sont séparées.

Des espaces comme YUV, YIQ, ou YCbCr séparent la composante de luminance (Y) des composantes de chrominance (U et V).

\Manip Convertir l'image RVB dans l'espace YUV par l'instruction suivante :

\begin{lstlisting}
image_yuv = cv2.cvtColor(image_rgb, cv2.COLOR_RGB2YUV)
\end{lstlisting}

\Manip Comparer alors l'image en niveau de gris obtenue par la méthode de moyennage pondérée et le canal Y de cette conversion.

\Quest Que pouvez-vous conclure sur la méthode de calcul utiliser pour la luminance (Y) ?

\medskip

\noindent \rule{\linewidth}{1pt}

Il existe d'autres espaces colorimétriques dans le domaine numérique. Voici un résumé non exhaustif :

\begin{center}
\begin{tabular}{|c|c|}
\hline
\textbf{Espace colorimétrique} & \textbf{Avantages} \\ \hline
\textbf{RGB} & Simple, utilisé pour les écrans et le rendu des couleurs. \\ \hline
\textbf{HSV / HSL} & Intuitif pour manipuler la couleur (teinte, saturation). \\ \hline
\textbf{YUV / YCbCr} & Sépare luminance et chrominance. \\ \hline
\textbf{CIE-Lab} & Uniformité perceptuelle, idéal pour mesurer les différences de couleur. \\ \hline
\textbf{CMY(K)} & Optimisé pour l'impression. \\ \hline
\textbf{XYZ} & Modèle basé sur la perception humaine. \\ \hline
\end{tabular}
\end{center}



\newpage
%%%%%%%%%%%%%%%%%%%%%%%%%%%%%%%%%%%%%%%%%%%%%%%%
%%%%%%%%%%%%%    Etape par étape

\section{Banc VI - Calibration}


%%%%%%%%%%%%%    Etape A
\subsection{VI-A - Calibrer la chaîne pour les cubes de couleur {[40 min]}}

%%%%%%%%%%%%%    Etape B
\subsection{VI-B - Calibrer la chaîne pour les formes colorés {[40 min]}}

%%%%%%%%%%%%%    Etape C
\subsection{VI-C - Calibrer la chaîne pour des mesures de distance {[40 min]}}

%%%%%%%%%%%%%%%%%%%%%%%%%%%%%%%%%%%%%%%%%%%%%%%%
%%% RESSOURCES COMPLEMENTAIRES		

%\newpage
%\begin{center}
%	\begin{minipage}{2.5cm}
%	\begin{center}
%		\includegraphics[width=5cm]{../images/Logo-LEnsE.png}
%	\end{center}
%\end{minipage}\hfill
%\begin{minipage}{10cm}
%	\begin{center}
%	\textbf{Institut d'Optique Graduate School }\\[0.1cm]
%    \textbf{Interfaçage Numérique}
%
%
%	\end{center}
%\end{minipage}\hfill
%
%
%\vspace{2cm}
%
%
%{\Large \bfseries \textsc{Interfaçage Numérique}} \\[0.5cm]
%{\large \bfseries Travaux Pratiques} \\[0.2cm]
%Semestre 6
%
%\vspace{1cm}
%
%% Title
%\rule{\linewidth}{0.4mm} \\[0.4cm]
%{ \Large \bfseries\color{violet_iogs} Ressources \\[0.4cm] }
%\rule{\linewidth}{0.4mm} \\[1cm]
%{\large Bloc Vision Industrielle}
%
%\end{center}
%
%\vspace{3cm}
%
%\textbf{\large Liste des ressources}
%\begin{itemize}
%	\item ???
%\end{itemize}
%
%\vfill



%
%\includepdf[pages={1,4}, nup=1x2, pagecommand={\section{\texorpdfstring{\hspace{-1em}}{Image Processing}}}\label{doc:image_proc}]{../docs/Image_Processing.pdf}
%\includepdf[pages={5,8,12,14,16,17,18,20,21,22,23,24,25,31}, nup=1x2]{../docs/Image_Processing.pdf}

\end{document}


