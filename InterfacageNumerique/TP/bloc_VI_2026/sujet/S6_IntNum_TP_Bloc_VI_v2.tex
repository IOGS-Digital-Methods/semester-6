%%%%%%%%%%%%%%%%%%%%%%%%%%%%%%%%%%%%%%%%%%
% Engineering problems / LaTeX Template
%		Semester 6
%		Institut d'Optique Graduate School
%%%%%%%%%%%%%%%%%%%%%%%%%%%%%%%%%%%%%%%%%%
%	6N-IntNum-BlocVI
%%%%%%%%%%%%%%%%%%%%%%%%%%%%%%%%%%%%%%%%%%
%
% Created by:
%	Julien VILLEMEJANE - 20/nov/2025
%	
%
%%%%%%%%%%%%%%%%%%%%%%%%%%%%%%%%%%%%%%%%%%
% Professional Newsletter Template
% LaTeX Template
% Version 1.0 (09/03/14)
%
% Created by:
% Bob Kerstetter (https://www.tug.org/texshowcase/) and extensively modified by:
% Vel (vel@latextemplates.com)
% 
% This template has been downloaded from:
% http://www.LaTeXTemplates.com
%
% License:
% CC BY-NC-SA 3.0 (http://creativecommons.org/licenses/by-nc-sa/3.0/)
%
%%%%%%%%%%%%%%%%%%%%%%%%%%%%%%%%%%%%%%%%%

\documentclass[a4paper,11pt,titlepage]{article} % The default font size is 10pt; 11pt and 12pt are alternatives

%%%%%%%%%%%%%%%%%%%%%%%%%%%%%%%%%%%%%%%%%%%%%%%%%%%%%%%%%%%%%%%%%%%%%%%%%%%%%%%%%%%%%%%%%%%%%%%%%%%%%%%%%%%%%%%%%%%%%%%%%%%%%%%%%%%%%%%%%%%%%%%%%%%%%%%%%%%%%%%%%%%%%%%%%%%%%%%%%%%%%%%%%%%%%%%%%%%%%%%%%%%%%%%%%%%%%%%%%%%%%%%%%%%%%%%%%%%%%%%%%%%%%%%%%%%%
\usepackage{opto_elec_villemejane}

%%%%%%%%%%%%%%%%%%%%%%%%%%%%%%%%%%%%%%%%%%%%%%%%
%%%%%%%%%%%%%%%%%%%%%%%%%%%%%%%%%%%%%%%%%%%%%%%%
\begin{document}


% Page de garde
\begin{titlepage}

\begin{center}
	\begin{minipage}{2.5cm}
	\begin{center}
		\includegraphics[width=8cm]{images/Logo-LEnsE.png}
	\end{center}
\end{minipage}\hfill
\begin{minipage}{10cm}
	\begin{center}
	\textbf{Institut d'Optique Graduate School }\\[0.1cm]
    \textbf{Interfaçage Numérique}


	\end{center}
\end{minipage}\hfill


\vspace{3cm}


{\huge \bfseries \textsc{Interfaçage Numérique}} \\[0.5cm]
{\large \bfseries Travaux Pratiques} \\[0.2cm]
Semestre 6

\vspace{1.5cm}
% Title
\rule{\linewidth}{0.3mm} \\[0.4cm]
{ \huge \bfseries\color{violet_iogs} Vision Industrielle \\[0.4cm] }
\rule{\linewidth}{0.3mm} \\[0.2cm]
{ \large \bfseries\color{violet_iogs} De la physique de la chaîne d'acquisition \\ à l'exploitation logicielle}
\rule{\linewidth}{0.3mm} \\[1cm]

4 séances

\bigskip

\begin{center}
	\includegraphics[width=0.8\textwidth]{images/vi_intro.png}
\end{center}

\vfill

\textit{Ce sujet est disponible au format électronique sur le site du LEnsE - https://lense.institutoptique.fr/ dans la rubrique Année / Première Année / Interfaçage Numérique S6 / Bloc 2 Vision Industrielle.}


% Bottom of the page
%{\textbf{\large {Année universitaire} 2024-2025}}

\end{center}
\end{titlepage}

\newpage
\strut % empty page

\newpage
\pagestyle{empty}

\begin{minipage}[c]{.25\linewidth}
	\includegraphics[width=5cm]{images/Logo-LEnsE.png}
\end{minipage} \hfill
\begin{minipage}[c]{.4\linewidth}

\begin{center}
\vspace{0.3cm}
{\Large \textsc{Interfaçage Numérique}}

\medskip

6N-047-SCI \qquad \textbf{\large Bloc VI}

\end{center}
\end{minipage}\hfill

\vspace{0.5cm}

\noindent \rule{\linewidth}{1pt}

{\noindent\Large  \rule[-7pt]{0pt}{30pt} \textbf{Vision Industrielle}}

\noindent \rule{\linewidth}{1pt}

\bigskip 

%%%%%%%%%%%%%%%%%%%%%%%%%%%%%%%%%%%%%%%%%%%%%%%%
%%%%%%%%%%%%%    A A V

La \textbf{vision industrielle} est une technologie qui permet à des machines d'\textbf{analyser automatiquement des scènes} pour \textbf{contrôler}, \textbf{guider} ou \textbf{inspecter} des objets sur des processus de production. Elle repose sur l'utilisation de \textbf{caméras}, d'\textbf{optique}, d'\textbf{éclairages} spécifiques (ou contraints), de \textbf{capteurs} et d'algorithmes de \textbf{traitement d'image}. 

\begin{center}
	\includegraphics[width=0.5\textwidth]{images/vi_intro_2.png}
\end{center}

Elle a pour but de \textbf{prendre des décisions automatiques} (ou aider l'être humain dans sa prise de décision) vis-à-vis d'un (ou plusieurs) objet(s) dans une scène spécifique : détecter des défauts ou des irrégularités, compter ou trier..., en rejetant ou validant automatiquement des produits, tout en assurant une constance de la qualité et de la répétabilité des opérations. 


\begin{center}
	\includegraphics[width=0.6\textwidth]{images/vi_chaine.png}
\end{center}


\newpage
\textit{Ce sujet a été co-conçu par une équipe d'étudiant$\cdot$es - Joséphine BECHU, Justine GABRIEL et Paul CHENEAU - lors d'un projet DEPhI de 2A - 2025-2026 - et l'équipe pédagogique d'Interfaçage Numérique.}

\medskip

\section{Acquis d'Apprentissages Visés (AAV)}
%%%%%%%%%%%%%%%%%%%%%%%%%%%%%%%%%%%%%%%%%%%%%%%%
%%%%%%%%%%%%%    A A V

À la fin de cette série de 4 séances, les étudiant$\cdot$es seront capable de :

\begin{itemize}
	\item Décomposer et paramétrer une chaîne de vision industrielle complète (du capteur au traitement de l'image),
	\item Comprendre les compromis physiques et numériques de chaque maillon,
	\item Réaliser un prototype d'inspection ou de mesure simple (tri d'objets),
	\item Justifier leurs choix de configuration (résolution, focale, éclairage...)
\end{itemize}

\medskip

%%%%%%%%%%%%%%%%%%%%%%%%%%%%%%%%%%%%%%%%%%%%%%%%
%%%%%%%%%%%%%    Ressources

\section{Ressources}

Un tutoriel sur les bases d'OpenCV est disponible à l’adresse suivante : 

\href{https://iogs-lense-training.github.io/image-processing/}{https://iogs-lense-training.github.io/image-processing/}

Un \textbf{kit d'images} est disponible sur le site du LEnsE dans la rubrique  \textit{Année / Première Année / Interfaçage Numérique S6 / Bloc 2 Caméra, Images et Interfaces / Images et OpenCV / Kit d'images}. 

Des \textbf{fichiers de fonctions} sont disponibles sur le site du LEnsE dans la rubrique \textit{Année / Première Année / Interfaçage Numérique S6 / Bloc 2 Caméra, Images et Interfaces / Images et OpenCV / Répertoire vers codes à tester}. 


Quelques \textbf{exemples} et explications sur les différents pré-traitements d'images est disponible sur le site du LEnsE dans la rubrique \textit{Année / Première Année / Interfaçage Numérique S6 / Bloc 2 Caméra, Images et Interfaces / Images et OpenCV / Image Processing with OpenCV}. 


\newpage
%%%%%%%%%%%%%%%%%%%%%%%%%%%%%%%%%%%%%%%%%%%%%%%%
%%%%%%%%%%%%%    Déroulement

\section{Déroulement}

Les sujets \textbf{TP1} et \textbf{TP2} se font en binôme et sont interchangeables (4 binômes commenceront par la TP1 lors de la première séance et les 4 autres binômes commenceront par le TP2).

Les sujets \textbf{TP3} et \textbf{TP4} se font par groupe de 4 étudiant$\cdot$es.


\subsection{TP1 - Banc de vision industrielle}

Le \textbf{TP1} se fera sur un banc de vision industrielle simple incluant une caméra, un objectif (focale : xx mm), un éclairage Effilux Ring-RGB et un ordinateur.

\medskip

\begin{itemize}
	\item {[20']} Prendre en main l'interface - caméra (temps d'intégration, histogramme) + éclairage RGB
	\item {[20']} Tester les outils de base proposés dans l'interface (coupe, lissage, seuillage)
	\item {[20']} Vérifier l'uniformité de l'éclairage sur objet uniforme - coupe dans l'image
	\item {[30']} Valider la linéarité de la caméra sur objet uniforme - pour différents temps d'intégration (en blanc et R/G/B)
	\item {[60']} Contrôler le champ de vision du système et sa résolution spatiale 
	\begin{itemize}
		\item placer une règle pour mesurer le FOV (Field of View)
		\item placer une mire pour mesurer la plus petite taille résoluble
		\item Modifier la profondeur binaire (8, 10, 12 bits) et voir l'impact sur la résolution spatiale / sur la qualité de l'image
		\item Revenir aux caractéristiques de l'objectif optique
	\end{itemize}
	\item {[30']} Contrôler le contraste du système en plaçant des mires en fonction du temps d'intégration, de l'éclairage
	\item {[60']} Analyser l'impact des propriétés des objets et des sources sur la valeur mesurée par la caméra
	\begin{itemize}
		\item Étudier les réflectances des cubes et le spectre des sources (fournis)
		\item Comparer les niveaux de gris obtenus sous différents éclairages des différents objets mis à disposition (cubes, formes...)
	\end{itemize}
	
\end{itemize}

\subsection{TP2 - Manipulations de base sous OpenCV}

Codes fournis (à modifier ?)

\begin{itemize}
	\item {[40']} Ouvrir une image - différence Gray/RGB
	\item {[20']} Calculer l'histogramme d'une image
	\item {[20']} Améliorer numériquement une image / contraste-luminosité
	\item {[20']} Appliquer un seuillage sur une image
	\item {[30']} Appliquer une érosion et une dilatation sur une image
	\item {[20']} Appliquer une ouverture et une fermeture sur une image
	\item {[20']} Appliquer un gradient sur une image
	\item {[30']} Calculer la FFT d'une image
	\item {[20']} Appliquer un filtre moyenneur sur une image
	\item {[20']} Appliquer un filtre passe-haut - Sobel
	
\end{itemize}


\newpage
\subsection{TP3 - Manipulations avancées sous OpenCV / Limites du banc de vision}

Sous OpenCV :

\begin{itemize}
	\item {[40']} Détecter des formes 
	\item {[40']} Détecter des couleurs 
	\item {[40']} Recolorisation (matrice de changement de base)
	
\end{itemize}

Segmentation de l'image ?

\medskip

Sur le banc de VI :

\begin{itemize}
	\item {[40']} Calibration de la chaîne sur les cubes de couleurs (couleur)
	\item {[40']} Calibration de la chaîne sur des formes colorées
	\item {[40']} Calibration de la chaîne pour des mesures de distance
\end{itemize}



\subsection{TP4 - Détection d'objets - Contrôle qualité}
	
Le but est à partir d'une série d'images (1 en RGB ou 3 en niveau de gris sous éclairage particulier) de détecter le nombre d'objets d'une forme et d'une couleur particulière et de valider si les pièces requises pour un assemblage sont bien présentes (simuler le packaging d'un produit fini - 6 pièces de forme et de couleurs distinctes par exemple).

La validation du packaging final devra se faire sans l'intervention de l'être humain mais un affichage propre des couleurs devra être fait pour le contrôle par le/la manipulateur$\cdot$trice.

Bonus : mesure de la conformité de la taille des pièces ?


%%%%%%%%%%%%%%%%%%%%%%%%%%%%%%%%%%%%%%%%%%%%%%%%
%%% RESSOURCES COMPLEMENTAIRES		

\newpage
\begin{center}
	\begin{minipage}{2.5cm}
	\begin{center}
		\includegraphics[width=5cm]{images/Logo-LEnsE.png}
	\end{center}
\end{minipage}\hfill
\begin{minipage}{10cm}
	\begin{center}
	\textbf{Institut d'Optique Graduate School }\\[0.1cm]
    \textbf{Interfaçage Numérique}


	\end{center}
\end{minipage}\hfill


\vspace{2cm}


{\Large \bfseries \textsc{Interfaçage Numérique}} \\[0.5cm]
{\large \bfseries Travaux Pratiques} \\[0.2cm]
Semestre 6

\vspace{1cm}

% Title
\rule{\linewidth}{0.4mm} \\[0.4cm]
{ \Large \bfseries\color{violet_iogs} Ressources \\[0.4cm] }
\rule{\linewidth}{0.4mm} \\[1cm]
{\large Bloc Vision Industrielle}

\end{center}

\vspace{3cm}

\textbf{\large Liste des ressources}
\begin{itemize}
	\item \hyperref[doc:image_proc]{Image Processing / Key concepts}
\end{itemize}

\vfill

\newpage
\strut % empty page


\includepdf[pages={1,4}, nup=1x2, pagecommand={\section{\texorpdfstring{\hspace{-1em}}{Image Processing}}}\label{doc:image_proc}]{../docs/Image_Processing.pdf}
\includepdf[pages={5,8,12,14,16,17,18,20,21,22,23,24,25,31}, nup=1x2]{../docs/Image_Processing.pdf}

\end{document}


