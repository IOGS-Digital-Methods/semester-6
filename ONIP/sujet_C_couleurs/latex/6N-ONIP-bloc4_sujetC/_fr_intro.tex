%----------------------------------------------------------------------------------------
%	SHORT INTRODUCTION
%----------------------------------------------------------------------------------------
Dans ce projet, on cherche à \textbf{calibrer une image à l'aide d'une mire ColorChecker SG}.
Le LEnsE dispose d'une mire ColorChecker SG (Spectral Gloss) fabriquée par Calibrite, c'est un outil de calibration des couleurs utilisé en imagerie numérique. Il se compose d'une série de petits carrés colorés. Ce type de mire est utilisé par exemple pour les photographies d'œuvre d'art.

%https://monappareilphotopro.fr/color-checker/.
%https://calibrite.com/fr/photo-target/?noredirect=fr-FR.
%https://www.lesnumeriques.com/accessoire-photo/x-rite-colorchecker-passport-p47385/test.html.

\medskip

D'un point de vue programmation, vous devrez développer ce projet selon les règles de la \textbf{programmation orientée objet}.
\textbf{Aucune fonction ne devra être utilisée en dehors d'un objet.}