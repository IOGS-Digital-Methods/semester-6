%----------------------------------------------------------------------------------------
%	STEP BY STEP
%----------------------------------------------------------------------------------------

\textit{Ce projet s'inscrit dans la continuité du cours d'optique instrumentale, vous pouvez tester vos simulations sur de travaux réalisés dans cette matière.}

\section{Introduction}

Le contexte de ce projet est l'optique dans les conditions de Gauß pour des systèmes centrés. En effet, on utilisera le formalisme matriciel (aussi appelé "matrice $ABCD$"), celui-ci est très commode pour la programmation informatique. Ce formalisme est équivalent à celui étudié en cours.

\medskip

Ce formalisme n'étant pas étudié en Optique Instrumentale, nous allons le présenter ici.


%%%%%%%%%%%%%%%%%%%%%%%
\section{Grandes étapes}

\begin{itemize}
	\item Définir une classe représentant les rayons.
	\item Définir des classes représentants des éléments optiques (lentilles, miroirs, dioptres, diaphragme).
	\item Définir une classe représentant un système optique. Cette classe pourra prendre en entrée une liste d'éléments optiques et une liste de rayons. Et permettra de propager ces derniers.
\end{itemize}

\medskip

\textit{Il est recommandé de surveiller visuellement l'avancée de votre projet à chaque étape. Pour cela, vous pouvez implémenter des méthodes (\texttt{plot}) dans vos classes.}
Une grande attention sera portée à la documentation de votre code et à la qualité des figures produites.

La programmation orientée objet est intéressante pour ce type de projet.
En effet, chaque élément optique peut être représenté par une classe.
Chaque élément optique pourra, par exemple, disposer d'une méthode pour propager les rayons : \texttt{propager\_rayon(self, rayon : Rayon)}.
Cette méthode pourra être appelée par la classe \texttt{SystemeOptique} pour propager les rayons dans le système, et ce, indépendamment de la nature des éléments optiques.

\section{Notions d'optique matricielle}
\textit{Les éléments suivants sont tirés du travail de F. Grillot.} \href{https://perso.telecom-paristech.fr/grillot/Page%20web_fichiers/optiquematricielle.pdf}{Lien du PDF sur l'optique matricielle} 
\textit{Il est également possible de se référer au cours d'Optique Instrumentale (S. de Rossi) de 1$^\text{re}$ année.}


On considère un système de coordonnées cartésiennes, où l'axe horizontal $z$ est l'axe optique, l'axe de révolution du système. La hauteur d'impact $y$ est mesurée par rapport à l'axe optique.

\begin{wrapfigure}{r}{0.38\textwidth}
	\centering
	\includegraphics[width=0.3\textwidth]{6N-ONIP-bloc4_sujetD/images/rayon.pdf}
	\caption{Représentation d'un rayon.}
	\label{fig:6N-ONIP-bloc4_sujetD_fig1}
\end{wrapfigure}

Un rayon est caractérisé par sa position et sa direction. Pour un plan de coordonné $z_0$, le rayon est caractérisé par un vecteur suivant :
$$p_{z_0} = \begin{pmatrix} y_0 \\ n\theta \end{pmatrix}_{z_0}$$

Pour effectuer des transformations sur les rayons, on utilise des matrices de transfert. La figure \ref{fig:6N-ONIP-bloc4_sujetD_fig1} illustre la propagation d'un rayon à travers un élément optique.

\textit{NB : Il existe des conventions différentes, par exemple sans les indices $n$ pour la deuxième composante, cela est moins intéressant. En effet, les déterminants des matrices de transfert ne sont plus égaux à 1, les matrices des dioptres plans ne sont plus égaux à la matrice identité...}

\subsection{Propagation d'un rayon}
Lorsqu'un se propage dans un milieu homogène, il est possible de déterminer sa position et sa direction après une distance $d$ par les relations suivantes :
$$\begin{pmatrix} y' \\ n\theta' \end{pmatrix}_{z+d} =
	\begin{pmatrix} 1 & d/n \\ 0 & 1 \end{pmatrix}
	\begin{pmatrix} y \\ n\theta \end{pmatrix}_z $$

Ainsi, on obtient la hauteur $y' = y + d\theta$ du rayon après une distance $d$. La direction du rayon est inchangée ($\theta' = \theta$).

\label{propagation}
\subsection{Propagation à travers des éléments optiques}
Pour pouvoir propager un rayon à travers un élément optique, il faut que le vecteur du rayon soit exprimé dans le plan de l'élément optique (voir partie \ref{propagation}).
Les valeurs en sortie sont indicées par $s$ pour \textit{sortie} et celles en entrée par $e$ pour \textit{entrée}. On définit les rayons de courbure $R = \overline{SC}$, où $S$ est le sommet de l'élément optique et $C$ le centre de courbure.

\subsubsection{Dioptre}
Lorsqu'un rayon rencontre un dioptre de rayon de courbure $R$, en passant d'un milieu d'indice $n_1$ à un milieu d'indice $n_2$. Il est possible de déterminer sa position et sa direction après le dioptre par les relations suivantes :
$$\begin{pmatrix} y_s \\ n_s\theta_s \end{pmatrix}_{\text{après}} =
	\begin{pmatrix} 1 & 0 \\ \dfrac{n_s-n_e}{R} & 1 \end{pmatrix}
	\begin{pmatrix} y_e \\ n_e\theta_e \end{pmatrix}_{\text{avant}}$$

La hauteur $y$ du rayon est inchangée ($y_s = y_e$) mais sa direction est modifiée.

\subsubsection{Miroir}
Lorsqu'un rayon est réfléchi par un miroir de rayon de courbure $R$, il est possible de déterminer sa position et sa direction après sa réflexion sur le miroir par les relations suivantes :
$$\begin{pmatrix} y_s \\ n_s\theta_s \end{pmatrix}_{\text{après}} =
	\begin{pmatrix} 1 & 0 \\ 2n/R & 1 \end{pmatrix}
	\begin{pmatrix} y_e \\ n_e\theta_e \end{pmatrix}_{\text{avant}}$$

Cette formule est un cas particulier de la formule pour un dioptre, où $n_1 = -n_2 = n$.

La hauteur $y$ du rayon est inchangée ($y_e = y_s$) mais sa direction est modifiée.
Pour le cas d'un miroir plan, $R = \infty$, le terme non diagonal de la matrice est nul.

\textit{Il faudrait veiller à être précautionneux sur le signe des indices (qui devient négatif après la réflexion sur un miroir), ainsi qu'au signe des rayons de courbure.}

\subsubsection{Lentille mince}

Lorsqu'un rayon traverse une lentille mince de distance focale $f'$, il est possible de déterminer sa position et sa direction après la lentille par les relations suivantes :
$$\begin{pmatrix} y_s \\ n_s\theta_s \end{pmatrix}_{\text{après}} =
	\begin{pmatrix} 1 & 0 \\ -1/f' & 1 \end{pmatrix}
	\begin{pmatrix} y_e \\ n_e\theta_e \end{pmatrix}_{\text{avant}}$$

À présent, la hauteur $y$ du rayon est inchangée ($y_e = y_s$) mais sa direction est modifiée. \textbf{Cet élément est le plus simple, il est recommandé de commencer l'implémentation par celui-ci.}

\subsubsection{Diaphragme}
Un diaphragme est un élément optique qui bloque les rayons qui ont une incidence trop élevée. Pour le modéliser, on propage le rayon jusqu'a la position du diaphragme et on bloque les rayons qui ont une incidence trop élevée.

\subsection{Exemples}

\subsubsection{Système à lentilles}
On considère le microscope issu du TD numéro 4 d'Optique Instrumentale (disponible sur le site du \href{https://lense.institutoptique.fr/optique-instrumentale-s5/}{LEnsE}).
Le système optique est composé d'un objectif de distance focale $f' = 40$\,mm et d'un oculaire de distance focale $f' = 12.5$\,mm. Un schéma réalisé via Python est donné en figure \ref{fig:6N-ONIP-bloc4_sujetD_fig2}.

\begin{figure}[h]
	\centering
	\includegraphics[width=0.75\textwidth]{\dirName /images/optics_matrix.pdf}
	\caption{Schéma du microscope du TD d'Optique Instrumentale.}
	\label{fig:6N-ONIP-bloc4_sujetD_fig2}
\end{figure}

Pour tracer un rayon rouge, qui est défini initialement dans le plan $z=0$ par $p = \left(0, n_\text{air} \theta \right)^T$, on a séquentiellement multiplié (par la gauche) les matrices de transfert suivantes :
\begin{itemize}
	\item Propagation dans l'air sur une distance de $d_1$ : $$M_1 = \begin{pmatrix}
			      1 & d_1/n_\text{air} \\
			      0 & 1
		      \end{pmatrix}$$
	\item Propagation à travers l'objectif, ici assimilé à une lentille mince : $$M_2 = \begin{pmatrix}
			      1       & 0 \\
			      -1/f_1' & 1
		      \end{pmatrix}$$
	\item Propagation à travers le diaphragme, on propage le rayon jusqu'a la position du diaphragme et on bloque les rayons qui ont une incidence trop élevée. $$M_3 = \begin{pmatrix}
			      1 & d_2/n_\text{air} \\
			      0 & 1
		      \end{pmatrix}$$
	\item Propagation jusqu'à l'oculaire : $$M_4 = \begin{pmatrix}
			      1 & d_3/n_\text{air} \\
			      0 & 1
		      \end{pmatrix}$$
	\item Propagation à travers l'oculaire, ici assimilé à une lentille mince : $$M_5 = \begin{pmatrix}
			      1       & 0 \\
			      -1/f_2' & 1
		      \end{pmatrix}$$
	\item Propagation jusqu'à la pupille de sortie : $$M_6 = \begin{pmatrix}
			      1 & d_4/n_\text{air} \\
			      0 & 1
		      \end{pmatrix}$$

\end{itemize}

Ce calcul est effectué dans la classe système optique après avoir défini les éléments optiques et les rayons que l'on souhaite propager.

Chaque rayon initial va donner naissance à de nouveaux rayons au cours de sa propagation. Il est donc nécessaire de stocker les rayons à chaque étape pour pouvoir les tracer à la fin.
\label{microscope}

\subsubsection{Système à miroirs}
On considère ici le \href{https://fr.wikipedia.org/wiki/Télescope_de_type_Cassegrain}{télescope de Cassegrain} issu du TD N$^o$8 d'optique instrumentale, représenté en figure \ref{fig:6N-ONIP-bloc4_sujetD_fig3}. Le système optique est composé d'un miroir primaire concave de rayon de courbure $R_1 = 250$\,mm et d'un miroir secondaire convexe de rayon de courbure $R_2 = 100$\,mm. Les autres grandeurs du système sont présentent dans le sujet de TD sur le site du \href{https://lense.institutoptique.fr/optique-instrumentale-s5/}{LEnsE}.

\begin{figure}[h]
	\centering
	\includegraphics[width=0.75\textwidth]{\dirName /images/cassegrain.pdf}
	\caption{Schéma du télescope de Cassegrain du TD d'Optique Instrumentale.}
	\label{fig:6N-ONIP-bloc4_sujetD_fig3}
\end{figure}

On a ici effectué le même type de calcul que pour le microscope, mais en adaptant les matrices de transfert aux éléments optiques du télescope de Cassegrain. On a ainsi multiplié les matrices de transfert suivantes :
$$\begin{pmatrix}
		y_s \\
		n_\text{air}\theta_s
	\end{pmatrix} =
	\begin{pmatrix}
		1 & d_3/n_\text{air} \\
		0 & 1
	\end{pmatrix}
	\times
	\begin{pmatrix}
		1                 & 0 \\
		2n_\text{air}/R_2 & 1
	\end{pmatrix}
	\times
	\begin{pmatrix}
		1 & -d_2/n_\text{air} \\
		0 & 1
	\end{pmatrix}
	\times
	\begin{pmatrix}
		1                 & 0 \\
		2n_\text{air}/R_1 & 1
	\end{pmatrix}
	\times
	\begin{pmatrix}
		1 & d_1/n_\text{air} \\
		0 & 1
	\end{pmatrix}
	\times
	\begin{pmatrix}
		y_e \\
		n_\text{air}\theta_e
	\end{pmatrix}$$
On remarque qu'après la réflexion sur un miroir, la direction du rayon est inversée, cela se traduit pas une inversion du signe de l'indice optique.

Sur la figure \ref{fig:6N-ONIP-bloc4_sujetD_fig3}, nous avons pu determiner la position de l'image ainsi que la position et la taille de la pupille de sortie.

\subsection{Grandeurs caractéristiques}

Un ensemble d'éléments optiques est caractérisé par sa matrice de transfert. Cette matrice est le produit des matrices de transfert de chaque élément optique, entre le plan d'entrée et le plan de sortie du système.

Par exemple, dans le cas du microscope de la partie \ref{microscope}, la matrice de transfert du système optique est donnée par la multiplication des matrices (dans le bon ordre) : $M_\text{SO} = M_5  M_4  M_3  M_2$.
Dans le cas général, on obtient une matrice

\begin{wrapfigure}{r}{0.45\textwidth}
	\centering
	\includegraphics[width=0.35\textwidth]{6N-ONIP-bloc4_sujetD/images/schema_grandeur.pdf}
\end{wrapfigure}

$$ M = \begin{pmatrix}
		A & B \\
		C & D
	\end{pmatrix}$$

On peut montrer les résultats suivants :
\begin{itemize}
	\item La vergence du système est donnée via $C$ par :
	      $$ V = -C = \dfrac{n_s}{f'} = -\dfrac{n_e}{f} $$
	      On peut donc en déduire la focale du système optique.
	\item Les positions des foyers principaux sont données via $D$ et $A$ par :
	      $$ \overline{EF} = fD \text{    et    }  \overline{SF} = f'A$$

	\item Les positions des plans principaux sont données via $D$ et $A$ par : $$ \overline{EP_1} = f(D-1) \text{    et    }  \overline{SP_2} = f'(A-1)$$

\end{itemize}
\label{grandeurs}


\section{Ouverture}
\textbf{Vous devrez choisir et réaliser au moins l'une des ouvertures suivantes dans le cadre de ce projet}.

\begin{description}

	\item[Ouverture A] \textbf{Conjugaison des pupilles}

	      \begin{itemize}
		      \item Vous pouvez faire en sorte de conjuguer des pupilles dans les différents espaces du système optique, afin d'obtenir la pupille de sortie ou la pupille d'entrée.
	      \end{itemize}

	      \qquad

	      % \item[Ouverture B] \textbf{Chromatisme}

	      %       \begin{itemize}
	      % 	      \item Vous pouvez faire en sorte de conjuguer des pupilles dans les différents espaces du système optique, afin d'obtenir la pupille de sortie ou la pupille d'entrée.
	      %       \end{itemize}

	      %       \qquad

	\item[Ouverture B] \textbf{Obtention des gradeurs caractéristiques}


	      \begin{itemize}
		      \item À l'aide de la partie \ref{grandeurs}, vous pouvez obtenir les grandeurs caractéristiques du système optique ainsi que les tracer leur position sur le graphique.
	      \end{itemize}

	      \qquad

	\item[Ouverture C] \textbf{Optimisation du système optique}
	      \begin{itemize}
		      \item Vous pouvez optimiser le système optique via des algorithmes d'optimisation (comme ceux de la bibliothèque SciPy). Par exemple, en faisant varier des grandeurs (distances focales, épaisseurs d'air...) afin d'obtenir une longueur focale et/ou un encombrement minimal.
	      \end{itemize}

\end{description}