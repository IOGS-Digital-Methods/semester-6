%%%%%%%%%%%%%%%%%%%%%%%%%%%%%%%%%%%%%%%%%%
% Engineering problems / LaTeX Template
%		Semester 6
%		Institut d'Optique Graduate School
%%%%%%%%%%%%%%%%%%%%%%%%%%%%%%%%%%%%%%%%%%
%	6N-ONIP-Block4	/ Project A - LEDs sources
%%%%%%%%%%%%%%%%%%%%%%%%%%%%%%%%%%%%%%%%%%
%
% Created by:
%	Julien VILLEMEJANE - 22/nov/2023
% Modified by:
%	
%
%%%%%%%%%%%%%%%%%%%%%%%%%%%%%%%%%%%%%%%%%%
% Professional Newsletter Template
% LaTeX Template
% Version 1.0 (09/03/14)
%
% Created by:
% Bob Kerstetter (https://www.tug.org/texshowcase/) and extensively modified by:
% Vel (vel@latextemplates.com)
% 
% This template has been downloaded from:
% http://www.LaTeXTemplates.com
%
% License:
% CC BY-NC-SA 3.0 (http://creativecommons.org/licenses/by-nc-sa/3.0/)
%
%%%%%%%%%%%%%%%%%%%%%%%%%%%%%%%%%%%%%%%%%

\documentclass[10pt]{article} % The default font size is 10pt; 11pt and 12pt are alternatives

\input{../../_latex_assets/5N_ONIP_structure.tex} % Include the document which specifies all packages and structural customizations for this template

\def\dirName{6N-ONIP-bloc4_intro}

\begin{document}


%----------------------------------------------------------------------------------------
%	DOCUMENT INFORMATIONS
%----------------------------------------------------------------------------------------
%----------------------------------------------------------------------------------------
%	DOCUMENT INFORMATIONS
%----------------------------------------------------------------------------------------
\def\module{Outils Numériques\\pour l'Ingénieur$\cdot$e\\en Physique}
\def\submodule{Outils Numériques}
\def\moduleSmall{6N-099-PHY / ONIP-2}
\def\year{2023-2024 / FISA}
\def\problem{Objets / Projet}
\def\problemName{Réalisation de cartes d'éclairement}

\def\validation{100\%}

\def\scheduleCM{0}
\def\scheduleTD{0}
\def\scheduleTDcomputer{4}
\def\scheduleTP{0}

\def\workingTeam{Par binôme}

\def\workingSpecial{}

\def\keywords{Photométrie; Carte d'éclairement; Modélisation de sources lumineuses}


%----------------------------------------------------------------------------------------
%	HEADER IMAGE
%----------------------------------------------------------------------------------------

\begin{figure}[H]
\centering\includegraphics[width=0.5\linewidth]{../../_latex_assets/logo_iogs.png}
\end{figure}

%----------------------------------------------------------------------------------------
%	SIDEBAR - FIRST PAGE
%----------------------------------------------------------------------------------------

\begin{minipage}[t]{.35\linewidth} % Mini page taking up 30% of the actual page
\begin{mdframed}[style=sidebar,frametitle={\module}] % Sidebar box

%-----------------------------------------------------------
%	DOCUMENT DESCRIPTION
\begin{center}

\textit{\large \centering \year}
\end{center}


\centerline {\rule{.70\linewidth}{.25pt}} % Horizontal line

\begin{center}
	\textit{\large \moduleSmall}
\end{center}

\centerline {\rule{.70\linewidth}{.25pt}} % Horizontal line

\begin{center}
	\textbf{\problem} ( \validation )
\end{center}

\centerline {\rule{.70\linewidth}{.25pt}} % Horizontal line

%-----------------------------------------------------------

\textbf{Concepts étudiés}

\begin{itemize}
%----------------------------------------------------------------------------------------
%	COVERED CONCEPTS
%----------------------------------------------------------------------------------------
\item[\textsc{\scriptsize [Phys]}] Photométrie
\item[\textsc{\scriptsize [Phys]}] Simulation éclairement
\item[\textsc{\scriptsize [Phys]}] Modélisation source ponctuelle
\item[\textsc{\scriptsize [Num]}] Affichage 2D et 3D
\end{itemize}

\centerline {\rule{.70\linewidth}{.25pt}} % Horizontal line

%-----------------------------------------------------------

\textbf{Mots clefs}

\keywords

\centerline {\rule{.70\linewidth}{.25pt}} % Horizontal line

%-----------------------------------------------------------

\textbf{Sessions}

\begin{itemize}
\item[\textbf{\scheduleCM}] Cours(s) - 1h30
\item[\textbf{\scheduleTD}] TD(s) - 1h30
\item[\textbf{\scheduleTDcomputer}] TD(s) Machine - 2h00
\item[\textbf{\scheduleTP}] TP(s) - 4h30
\end{itemize}

\centerline {\rule{.70\linewidth}{.25pt}} % Horizontal line

{\large Travail}

\textbf{\workingTeam}

\textbf{\workingSpecial}


%-----------------------------------------------------------

\end{mdframed}


\centering
\begin{minipage}[t]{.95\linewidth}
\textbf{Institut d'Optique}\\
Graduate School, \textit{France}\\
\href{https://www.institutoptique.fr}{https://www.institutoptique.fr}

\medskip
\textbf{GitHub - Digital Methods}

\href{https://github.com/IOGS-Digital-Methods}{https://github.com/IOGS-Digital-Methods}

\end{minipage}

\end{minipage}\hfill % End the sidebar mini page 
%
%----------------------------------------------------------------------------------------
%	MAIN BODY - FIRST PAGE
%----------------------------------------------------------------------------------------
%
\begin{minipage}[t]{.60\linewidth} % Mini page taking up 66% of the actual page

\hypertarget{context}{\heading{\huge \problemName}{6pt}} % \hypertarget provides a label to reference using \hyperlink{label}{link text}

\centerline {\rule{.70\linewidth}{.25pt}} % Horizontal line

%% Short introduction 
%----------------------------------------------------------------------------------------
%	SHORT INTRODUCTION
%----------------------------------------------------------------------------------------
Dans ce projet, on se propose de simuler des tracés de rayons dans des systèmes optiques centrés.
Le principal intérêt est de pouvoir dimensionner un système optique, dans la continuité du cours d'Optique Instrumentale.
On ne s'intéresse pas aux problèmatiques d'aberrations qui serons abordés dans le cours de Conception des Systèmes Optiques en 2\ieme{} année.

\medskip

D'un point de vue programmation, vous devrez développer ce projet selon les règles de la \textbf{programmation orientée objet}.

\textbf{Aucune fonction ne devra être utilisée en dehors d'un objet.}


%%

\bigskip

%----------------------------------------------------------------------------------------
%	progress session by session
%----------------------------------------------------------------------------------------

\begin{mdframed}[style=intextbox,frametitle={Déroulement du module}] % Sidebar box

Ce module se déroule sur \textbf{6 séances} :

\begin{description}
	\item[Séance 1] Découverte de la programmation objet 
	\item[Séances 2 à 5] Réalisation du mini-projet en binôme
	\item[Séance 6] Evaluation du mini-projet en binôme
\end{description}


\end{mdframed}

%%

\bigskip

%----------------------------------------------------------------------------------------
%	IN-TEXT BOX / Deliverables
%----------------------------------------------------------------------------------------


\begin{mdframed}[style=intextbox,frametitle={Livrables attendus}] % Sidebar box

%----------------------------------------------------------------------------------------
%	DEVELIRABLES
%----------------------------------------------------------------------------------------

Afin de faciliter la réalisation du mini-projet proposé, nous vous suggérons tout au long du développement de mettre à jour les documents suivants :

\begin{enumerate}
\item \textbf{Diagramme de classe} et répartition du travail
\item \textbf{Classes commentées} (selon la norme PEP 8) pour générer des objets 
\item \textbf{Graphiques légendés} incluant toutes les données nécessaires à la bonne compréhension des données présentées
\item \textbf{Analyse des figures} obtenues 
\end{enumerate}




\end{mdframed}



%\begin{wrapfigure}[7]{l}[0pt]{0pt} % In-line figure with text wrapping around it
%\includegraphics[width=0.3\textwidth]{engPb_S5_01/placeholder.jpg}
%\end{wrapfigure}

\end{minipage} % End the main body - first page mini page

%-----------------------------------------------------------

\hypertarget{ressources}{\heading{Ressources}{6pt}} % \hypertarget provides a label to reference using \hyperlink{label}{link text}

%----------------------------------------------------------------------------------------
%	RESSOURCES
%----------------------------------------------------------------------------------------
Cette séquence est basée sur le langage Python. Vous pouvez utiliser l'environnement \textbf{Pycharm}.
Des tutoriels Python (et sur les bibliothèques classiques : Numpy, Matplotlib or Scipy) sont disponibles à l'adresse : \href{http://lense.institutoptique.fr/python/}{http://lense.institutoptique.fr/python/}. 





%----------------------------------------------------------------------------------------
%	MAIN BODY - SECOND PAGE
%----------------------------------------------------------------------------------------

\begin{minipage}[t]{.66\linewidth} % Mini page taking up 66% of the actual page

%-----------------------------------------------------------

\hypertarget{stepbystep}{\heading{Séance 1 - Programmation orientée objet}{6pt}} % \hypertarget provides a label to reference using \hyperlink{label}{link text}

%----------------------------------------------------------------------------------------
%	STEP BY STEP
%----------------------------------------------------------------------------------------

Dans ce module, vous serez amenés à développer une application selon les \textbf{principes de la programmation orientée objet}.

Afin de vous familiarisez avec les principes de base, la première séance sera consacrée à \textbf{l'étude et la mise en oeuvre d'exemples de la programmation orientée objet} en Python : écriture d'une classe, instanciation d'un objet, interaction entre les objets.





\centerline {\rule{.70\linewidth}{.25pt}} % Horizontal line

%----------------------------------------------------------------------------------------
%	IN-TEXT BOX / Intended learning outcomes
%----------------------------------------------------------------------------------------

\begin{mdframed}[style=aavbox,frametitle={Acquis d'Apprentissage Visés}]

\begin{enumerate}
%----------------------------------------------------------------------------------------
%	Intended Learning Outcomes - Numerical Tools
%----------------------------------------------------------------------------------------
\item \textbf{Créer des classes} pour stocker et manipuler des données numériques.
\item \textbf{Définir et documenter les méthodes et attributs} de chaque classe
\item \textbf{Produire des figures} claires et légendées à partir de signaux numériques (image, couleur), incluant un titre, des axes, des légendes

\end{enumerate}


%\begin{center}
%{\large \textsc{Côté Physique}}
%\end{center}
%
%\begin{enumerate}
%%----------------------------------------------------------------------------------------
%	Intended Learning Outcomes - Physics
%----------------------------------------------------------------------------------------
\item \textbf{Modéliser une source ponctuelle} de lumière
\item \textbf{Réaliser une carte d'éclairement} pour N sources ponctuelles
%\end{enumerate}

\end{mdframed}
\medskip

L'ensemble des documents du module ONIP-2 se trouve sur le site du LEnsE : \hyperlink{http://lense.institutoptique.fr/ONIP/}{http://lense.institutoptique.fr/ONIP/} . Les exemples pour cette première séance se trouvent dans la rubrique \textbf{BLOC 4}.
Ajoutez \textit{from \_\_future\_\_ import annotations} en début de code pour une meilleure gestion des annotations.  \textbf{Attention}, les classes crées pour ces exercices ne seront pas réutilisées pour les projets. 
\medskip


\begin{mdframed}[style=intextbox,frametitle={Exercice 1 - Classe Point}] % Sidebar box

En vous inspirant de la classe \textbf{Animal} (simple) :

\begin{itemize}
	\item créez un nouveau fichier .py
	\item définissez une classe \textbf{Point}, permettant de modéliser un point dans un espace en 2 dimensions par ses coordonnées x et y
	\item instanciez deux objets de type Point avec des coordonnées différentes
	\item redéfinissez la méthode \textbf{\textit{\_\_str\_\_}} pour qu'elle affiche les coordonnées d'un objet de type \textbf{Point} (voir exemple de la classe \textbf{Animal})
	\item définissez une méthode barycentre qui prend un objet \textbf{Point} en argument et retourne un nouvel objet point correspondant au barycentre
	\item vérifiez vos différentes méthodes
\end{itemize}

\end{mdframed}

\medskip

\begin{mdframed}[style=intextbox,frametitle={Exercice 2 - Classe Rectangle}] % Sidebar box

Dans le fichier précédent et en utilisant la classe \textbf{Point} :

\begin{itemize}
	\item définissez une classe \textbf{Rectangle}, permettant de modéliser un rectangle orienté selon les axes x et y à partir de deux objets de type Point (sommets opposés du rectangle)
	\item définissez des méthodes \textit{\textbf{perimetre}} et \textit{\textbf{surface}} permettant de calculer le périmètre et la surface d'un objet de type \textbf{Rectangle}
    \item définissez une méthode recouvrement qui prend un objet \textbf{Rectangle} en argument et renvoie un booléen indiquant si il y a un recouvrement entre les deux rectangles.
    \item définir une méthode agrandir qui applique un facteur d'échelle au rectangle
    \item redéfinissez la méthode \textbf{\textit{\_\_str\_\_}}
	\item testez l'ensemble de vos méthodes sur différents objets de type \textbf{Rectangle}
\end{itemize}

\end{mdframed}


\medskip

\begin{mdframed}[style=intextbox,frametitle={Exercice 3 - Classe Cercle}] % Sidebar box

\begin{itemize}
    \item même exercice avec un cercle défini par son centre et un point du rayon. Cette fois, la méthode intersetion retournera une liste de zéro à deux objets points correspondants aux points d'interection.
    \item redéfinissez la méthode \textbf{\textit{\_\_str\_\_}}
\end{itemize}
\end{mdframed}



%----------------------------------------------------------------------------------------

\end{minipage}\hfill % End of the main body - second page mini page
\begin{minipage}[t]{.30\linewidth} % Mini page taking up 30% of the actual page

%----------------------------------------------------------------------------------------
%	SIDEBAR - SECOND PAGE
%----------------------------------------------------------------------------------------

\begin{mdframed}[style=sidebar,frametitle={}] % Sidebar box

\heading{Outils Numériques}{0pt}

\centerline {\rule{.40\linewidth}{.1pt}} % Horizontal line

\textbf{Fonctions et bibliothèques conseillées} :

%----------------------------------------------------------------------------------------
%	NUMERICAL TOOLS / BASICS
%----------------------------------------------------------------------------------------

\begin{itemize}
	\item \textbf{Numpy} gestion de matrices
	\item \textbf{Matplotlib} affichage de données
	\item \textbf{Scipy} fonctions scientifiques
\end{itemize}

\centerline {\rule{.40\linewidth}{.1pt}} % Horizontal line

%\textbf{Outils avancés} :
%
%\input{ \dirName /_fr_num_advanced.tex}

\end{mdframed}\hfill


\begin{mdframed}[style=sidebar,frametitle={}] % Sidebar box

\heading{Fichiers d'exemple}{0pt}

\centerline {\rule{.40\linewidth}{.1pt}} % Horizontal line

Classe \textbf{Animal} (simple) :

onip\_b4\_a\_classe\_simple.py

\medskip

Classe \textbf{Animal} (redéfinition str) :

onip\_b4\_b\_classe\_simple 
\_redefinition.py

\medskip

Classes \textbf{Dog} et \textbf{Cat} :

onip\_b4\_c\_classe\_heritage



\centerline {\rule{.40\linewidth}{.1pt}} % Horizontal line

%\textbf{Outils avancés} :
%
%\input{ \dirName /_fr_num_advanced.tex}

\end{mdframed}\hfill


%----------------------------------------------------------------------------------------

\end{minipage} % End of the sidebar mini page

%----------------------------------------------------------------------------------------

\newpage

\hypertarget{eval}{\heading{Evaluation du module}{6pt}} % \hypertarget provides a label to reference using \hyperlink{label}{link text}

Lors de la sixième séance, vous devrez présenter le travail que vous avez réalisé sur l'un des deux projets proposés.

Vous devrez également \textbf{au cours des séances} faire \textbf{valider l'application minimale} visée. Le descriptif des attendus de l'application minimale est donné dans les sujets des projets.

Vous devrez enfin présenter des résultats sur l'\textbf{une des ouvertures proposées} sur chacun des projets.

\bigskip

\bigskip

\begin{mdframed}[style=intextbox,frametitle={Présentation du travail}] % Sidebar box

Vous serez \textbf{convoqués par binôme} 15 min avant le début de votre présentation. 

\medskip 

Vous aurez alors \textbf{7 min} pour présenter les aspects suivants de votre travail :

\begin{description}
	\item[1 min] Présentation générale - Problématique
	\item[2 min] Résultats sur le système final 
	\item[4 min] Présentation de code
\end{description}

\medskip

Vous aurez ensuite 5 min de questions par le jury. Vous devrez être en mesure d'exécuter votre code pour montrer son bon fonctionnement. Votre code devra être déposé sur e-campus avant le jour de l'évaluation.

\end{mdframed}


\bigskip

\bigskip

\begin{mdframed}[style=intextbox,frametitle={Critères d'évaluation}] % Sidebar box

Vous serez évalué.e selon les critères suivants :

\begin{itemize}
	\item \textbf{\large Méthodologie}
	\begin{itemize}
		\item Bon usage de la programmation orientée objet
		\begin{itemize}
			\item objets mis en oeuvre
			\item attributs et méthodes utiles pour chaque objet
		\end{itemize}
		\item Répartition de l'écriture du code
	\end{itemize}

	\item \textbf{\large Programmation}
	\begin{itemize}
		\item Respect de la charte PEP8 (noms des variables, méthodes, commentaires...)
		\item Utilisation, écriture et validation de classes
	\end{itemize}

	\item \textbf{\large Physique}
	\begin{itemize}
		\item Graphiques pertinents et légendés
		\item Données pertinentes de test
	\end{itemize}

	\item \textbf{\large Avancement}
	\begin{itemize}
		\item Application de base validée
		\item Ouverture
	\end{itemize}

\end{itemize}

\end{mdframed}

\end{document} 