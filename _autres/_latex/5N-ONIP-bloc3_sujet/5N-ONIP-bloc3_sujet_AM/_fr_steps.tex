%----------------------------------------------------------------------------------------
%	STEP BY STEP
%----------------------------------------------------------------------------------------


Dans cette séquence, vous serez amenés à utiliser des données provenant d'un fichier de points issu d'un \textbf{oscilloscope}. Le fichier se nomme \mbox{\textsc{B3\_data\_01.csv}} (modulante sinusoïdale).

Le signal qu'il contient est un enregistrement d'une \textbf{transmission d'informations modulées en amplitude} par un signal porteur sinusoïdal.

\medskip

Deux autres fichiers vous sont également proposés :
\begin{itemize}
	\item  \mbox{\textsc{B3\_data\_02.txt}} contenant un signal sonore modulé en amplitude à déchiffrer...
	\begin{itemize}
		\item Format de données binaire 64 / Modulante sinusoïdale / Fichier sonore : 24 kHz / 16 bits
	\end{itemize}
	\item  \mbox{\textsc{B3\_data\_03.txt}} contenant un ensemble de signaux modulés en amplitude à l'aide de différentes porteuses.
	\begin{itemize}
		\item Format de données binaire 64 / Modulantes sinusoïdales / Fichier sonore : 160 kHz / 16 bits
	\end{itemize}
\end{itemize}



