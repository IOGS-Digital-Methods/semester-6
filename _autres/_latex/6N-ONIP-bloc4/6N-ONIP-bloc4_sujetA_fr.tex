%%%%%%%%%%%%%%%%%%%%%%%%%%%%%%%%%%%%%%%%%%
% Engineering problems / LaTeX Template
%		Semester 6
%		Institut d'Optique Graduate School
%%%%%%%%%%%%%%%%%%%%%%%%%%%%%%%%%%%%%%%%%%
%	6N-ONIP-Block4	/ Project A - LEDs sources
%%%%%%%%%%%%%%%%%%%%%%%%%%%%%%%%%%%%%%%%%%
%
% Created by:
%	Julien VILLEMEJANE - 22/nov/2023
% Modified by:
%	
%
%%%%%%%%%%%%%%%%%%%%%%%%%%%%%%%%%%%%%%%%%%
% Professional Newsletter Template
% LaTeX Template
% Version 1.0 (09/03/14)
%
% Created by:
% Bob Kerstetter (https://www.tug.org/texshowcase/) and extensively modified by:
% Vel (vel@latextemplates.com)
% 
% This template has been downloaded from:
% http://www.LaTeXTemplates.com
%
% License:
% CC BY-NC-SA 3.0 (http://creativecommons.org/licenses/by-nc-sa/3.0/)
%
%%%%%%%%%%%%%%%%%%%%%%%%%%%%%%%%%%%%%%%%%

\documentclass[10pt]{article} % The default font size is 10pt; 11pt and 12pt are alternatives

\input{../../_latex_assets/5N_ONIP_structure.tex} % Include the document which specifies all packages and structural customizations for this template

\def\dirName{6N-ONIP-bloc4_sujetA}

\begin{document}


%----------------------------------------------------------------------------------------
%	DOCUMENT INFORMATIONS
%----------------------------------------------------------------------------------------
%----------------------------------------------------------------------------------------
%	DOCUMENT INFORMATIONS
%----------------------------------------------------------------------------------------
\def\module{Outils Numériques\\pour l'Ingénieur$\cdot$e\\en Physique}
\def\submodule{Outils Numériques}
\def\moduleSmall{6N-099-PHY / ONIP-2}
\def\year{2023-2024 / FISA}
\def\problem{Objets / Projet}
\def\problemName{Réalisation de cartes d'éclairement}

\def\validation{100\%}

\def\scheduleCM{0}
\def\scheduleTD{0}
\def\scheduleTDcomputer{4}
\def\scheduleTP{0}

\def\workingTeam{Par binôme}

\def\workingSpecial{}

\def\keywords{Photométrie; Carte d'éclairement; Modélisation de sources lumineuses}


%----------------------------------------------------------------------------------------
%	HEADER IMAGE
%----------------------------------------------------------------------------------------

\begin{figure}[H]
\centering\includegraphics[width=0.5\linewidth]{../../_latex_assets/logo_iogs.png}
\end{figure}

%----------------------------------------------------------------------------------------
%	SIDEBAR - FIRST PAGE
%----------------------------------------------------------------------------------------

\begin{minipage}[t]{.35\linewidth} % Mini page taking up 30% of the actual page
\begin{mdframed}[style=sidebar,frametitle={\module}] % Sidebar box

%-----------------------------------------------------------
%	DOCUMENT DESCRIPTION
\begin{center}

\textit{\large \centering \year}
\end{center}


\centerline {\rule{.70\linewidth}{.25pt}} % Horizontal line

\begin{center}
	\textit{\large \moduleSmall}
\end{center}

\centerline {\rule{.70\linewidth}{.25pt}} % Horizontal line

\begin{center}
	\textbf{\problem} ( \validation )
\end{center}

\centerline {\rule{.70\linewidth}{.25pt}} % Horizontal line

%-----------------------------------------------------------

\textbf{Concepts étudiés}

\begin{itemize}
%----------------------------------------------------------------------------------------
%	COVERED CONCEPTS
%----------------------------------------------------------------------------------------
\item[\textsc{\scriptsize [Phys]}] Photométrie
\item[\textsc{\scriptsize [Phys]}] Simulation éclairement
\item[\textsc{\scriptsize [Phys]}] Modélisation source ponctuelle
\item[\textsc{\scriptsize [Num]}] Affichage 2D et 3D
\end{itemize}

\centerline {\rule{.70\linewidth}{.25pt}} % Horizontal line

%-----------------------------------------------------------

\textbf{Mots clefs}

\keywords

\centerline {\rule{.70\linewidth}{.25pt}} % Horizontal line

%-----------------------------------------------------------

\textbf{Sessions}

\begin{itemize}
\item[\textbf{\scheduleCM}] Cours(s) - 1h30
\item[\textbf{\scheduleTD}] TD(s) - 1h30
\item[\textbf{\scheduleTDcomputer}] TD(s) Machine - 2h00
\item[\textbf{\scheduleTP}] TP(s) - 4h30
\end{itemize}

\centerline {\rule{.70\linewidth}{.25pt}} % Horizontal line

{\large Travail}

\textbf{\workingTeam}

\textbf{\workingSpecial}


%-----------------------------------------------------------

\end{mdframed}


\centering
\begin{minipage}[t]{.95\linewidth}
\textbf{Institut d'Optique}\\
Graduate School, \textit{France}\\
\href{https://www.institutoptique.fr}{https://www.institutoptique.fr}

\medskip
\textbf{GitHub - Digital Methods}

\href{https://github.com/IOGS-Digital-Methods}{https://github.com/IOGS-Digital-Methods}

\end{minipage}

\end{minipage}\hfill % End the sidebar mini page 
%
%----------------------------------------------------------------------------------------
%	MAIN BODY - FIRST PAGE
%----------------------------------------------------------------------------------------
%
\begin{minipage}[t]{.60\linewidth} % Mini page taking up 66% of the actual page

\hypertarget{context}{\heading{\huge \problemName}{6pt}} % \hypertarget provides a label to reference using \hyperlink{label}{link text}

\centerline {\rule{.70\linewidth}{.25pt}} % Horizontal line

%% Short introduction 
%----------------------------------------------------------------------------------------
%	SHORT INTRODUCTION
%----------------------------------------------------------------------------------------
Dans ce projet, on se propose de simuler des tracés de rayons dans des systèmes optiques centrés.
Le principal intérêt est de pouvoir dimensionner un système optique, dans la continuité du cours d'Optique Instrumentale.
On ne s'intéresse pas aux problèmatiques d'aberrations qui serons abordés dans le cours de Conception des Systèmes Optiques en 2\ieme{} année.

\medskip

D'un point de vue programmation, vous devrez développer ce projet selon les règles de la \textbf{programmation orientée objet}.

\textbf{Aucune fonction ne devra être utilisée en dehors d'un objet.}

%%

\bigskip

%----------------------------------------------------------------------------------------
%	IN-TEXT BOX / Intended learning outcomes
%----------------------------------------------------------------------------------------

\begin{mdframed}[style=aavbox,frametitle={Acquis d'Apprentissage Visés}]

En résolvant ce problème, les étudiant$\cdot$e$\cdot$s seront capables de  :

\centerline {\rule{.40\linewidth}{.1pt}} % Horizontal line

\begin{center}
{\large \textsc{Côté Numérique}}
\end{center}

\begin{enumerate}
%----------------------------------------------------------------------------------------
%	Intended Learning Outcomes - Numerical Tools
%----------------------------------------------------------------------------------------
\item \textbf{Créer des classes} pour stocker et manipuler des données numériques.
\item \textbf{Définir et documenter les méthodes et attributs} de chaque classe
\item \textbf{Produire des figures} claires et légendées à partir de signaux numériques (image, couleur), incluant un titre, des axes, des légendes

\end{enumerate}

\centerline {\rule{.40\linewidth}{.1pt}} % Horizontal line

\begin{center}
{\large \textsc{Côté Physique}}
\end{center}

\begin{enumerate}
%----------------------------------------------------------------------------------------
%	Intended Learning Outcomes - Physics
%----------------------------------------------------------------------------------------
\item \textbf{Modéliser une source ponctuelle} de lumière
\item \textbf{Réaliser une carte d'éclairement} pour N sources ponctuelles
\end{enumerate}

\end{mdframed}
\medskip



%\begin{wrapfigure}[7]{l}[0pt]{0pt} % In-line figure with text wrapping around it
%\includegraphics[width=0.3\textwidth]{engPb_S5_01/placeholder.jpg}
%\end{wrapfigure}

\end{minipage} % End the main body - first page mini page

%----------------------------------------------------------------------------------------
%	MAIN BODY - SECOND PAGE
%----------------------------------------------------------------------------------------

\begin{minipage}[t]{.66\linewidth} % Mini page taking up 66% of the actual page

%----------------------------------------------------------------------------------------
%	IN-TEXT BOX / Deliverables
%----------------------------------------------------------------------------------------


\begin{mdframed}[style=intextbox,frametitle={Livrables attendus}] % Sidebar box

%----------------------------------------------------------------------------------------
%	DEVELIRABLES
%----------------------------------------------------------------------------------------

Afin de faciliter la réalisation du mini-projet proposé, nous vous suggérons tout au long du développement de mettre à jour les documents suivants :

\begin{enumerate}
\item \textbf{Diagramme de classe} et répartition du travail
\item \textbf{Classes commentées} (selon la norme PEP 8) pour générer des objets 
\item \textbf{Graphiques légendés} incluant toutes les données nécessaires à la bonne compréhension des données présentées
\item \textbf{Analyse des figures} obtenues 
\end{enumerate}




\end{mdframed}

%-----------------------------------------------------------

\hypertarget{stepbystep}{\heading{Programmation orientée objet}{6pt}} % \hypertarget provides a label to reference using \hyperlink{label}{link text}

%----------------------------------------------------------------------------------------
%	STEP BY STEP
%----------------------------------------------------------------------------------------

Dans ce module, vous serez amenés à développer une application selon les \textbf{principes de la programmation orientée objet}.

Afin de vous familiarisez avec les principes de base, la première séance sera consacrée à \textbf{l'étude et la mise en oeuvre d'exemples de la programmation orientée objet} en Python : écriture d'une classe, instanciation d'un objet, interaction entre les objets.





\centerline {\rule{.70\linewidth}{.25pt}} % Horizontal line

%-----------------------------------------------------------

\hypertarget{ressources}{\heading{Ressources}{6pt}} % \hypertarget provides a label to reference using \hyperlink{label}{link text}

%----------------------------------------------------------------------------------------
%	RESSOURCES
%----------------------------------------------------------------------------------------
Cette séquence est basée sur le langage Python. Vous pouvez utiliser l'environnement \textbf{Pycharm}.
Des tutoriels Python (et sur les bibliothèques classiques : Numpy, Matplotlib or Scipy) sont disponibles à l'adresse : \href{http://lense.institutoptique.fr/python/}{http://lense.institutoptique.fr/python/}. 



%----------------------------------------------------------------------------------------

\end{minipage}\hfill % End of the main body - second page mini page
\begin{minipage}[t]{.30\linewidth} % Mini page taking up 30% of the actual page

%----------------------------------------------------------------------------------------
%	SIDEBAR - SECOND PAGE
%----------------------------------------------------------------------------------------

\begin{mdframed}[style=sidebar,frametitle={}] % Sidebar box

\heading{Outils Numériques}{0pt}

\centerline {\rule{.40\linewidth}{.1pt}} % Horizontal line

\textbf{Fonctions et bibliothèques conseillées} :

%----------------------------------------------------------------------------------------
%	NUMERICAL TOOLS / BASICS
%----------------------------------------------------------------------------------------

\begin{itemize}
	\item \textbf{Numpy} gestion de matrices
	\item \textbf{Matplotlib} affichage de données
	\item \textbf{Scipy} fonctions scientifiques
\end{itemize}

\centerline {\rule{.40\linewidth}{.1pt}} % Horizontal line


\textbf{Outils avancés} :

\input{ \dirName /_fr_num_advanced.tex}

\end{mdframed}\hfill

%----------------------------------------------------------------------------------------

\end{minipage} % End of the sidebar mini page

%----------------------------------------------------------------------------------------
\newpage

\hypertarget{stepbystep}{\heading{Etapes}{6pt}}

\textit{Les représentations graphiques à produire à chacune des étapes seront accompagnées de \textbf{renseignements quantitatifs} comme la valeur moyenne de l'éclairement, son écart-type et son écart Pic à Vallée, absolus et relatifs, sur l'ensemble de la zone représentée (ou sur une sous-partie rectangulaire ou circulaire de celle-ci). Par ailleurs, une \textbf{représentation en 3D de la position et de l'orientation des N sources} sera utile.}

\qquad

\qquad

\begin{description}
	\item[Etape 1] \textbf{Carte d'éclairement pour une source ponctuelle - direction perpendiculaire par rapport au plan éclairé}
	
	\begin{itemize}
		\item Création d'une classe LED\_source
		\item Création d'une classe Carte
		\item Affichage 2D d'une carte avec une source ponctuelle
		\item Affichage position des sources sur la carte
		\item Validation du modèle
	\end{itemize}
	

\qquad
	
	\item[Etape 2] \textbf{Carte d'éclairement pour une source ponctuelle - direction quelconque par rapport au plan éclairé}
	
	\begin{itemize}
		\item Modification de la classe LED\_source
		\item Affichage 2D d'une carte avec une source ponctuelle
		\item Affichage position des sources sur la carte
		\item Validation du modèle
	\end{itemize}
	
\qquad

	\item[Etape 3] \textbf{Carte d'éclairement pour N sources ponctuelles - direction quelconque par rapport au plan éclairé}
	
	\begin{itemize}
		\item Modification de la classe LED\_source
		\item Affichage 2D d'une carte avec une source ponctuelle
		\item Affichage position des sources sur la carte
		\item Validation du modèle
	\end{itemize}

\qquad

	\item[Ouverture A] \textbf{Carte d'éclairement 3D}
	
	\begin{itemize}
		\item Affichage 3D d'une carte avec une source ponctuelle avec position des sources sur la carte
	\end{itemize}	

\qquad

	\item[Ouverture B] \textbf{Optimisation d'un éclairement}
	
	\begin{itemize}
		\item Optimisation du nombre et de l'orientation de LEDs pour obtenir un flux lumineux donné sur un plan de travail
	\end{itemize}	

\qquad
	
	\item[Ouverture C] \textbf{Réalisation d'une IHM / PyQt6}
	\begin{itemize}
		\item Utilisation de PyQt6 pour l'intégration des précédentes fonctions dans une IHM
		\item Intégration des graphiques avec pyqtgraph
		\item Possibilité d'ajouter des sources et d'afficher la contribution indépendante de chacune des sources		
	\end{itemize}
	
\end{description}

\newpage

\hypertarget{stepbystep}{\heading{Critères d'évaluation}{6pt}}

\textbf{Grille à simplifier (bilan Semestre 5)}

\begin{itemize}
	\item \textbf{METHODES NUMERIQUE}
	\begin{itemize}
		\item \textbf{Ecriture Matricielle / Vectorielle}
		\begin{itemize}
			\item utilisation des méthodes liées aux vecteurs/matrices (Numpy)
			\item aucune boucle \textbf{for} inutile
		\end{itemize}		 
		\item \textbf{Organisation en actions élémentaires}
		\begin{itemize}
			\item les étapes sont découpées en fonctionnalité plus simple à tester
		\end{itemize}
		\item \textbf{Description des tests de validation}
		\begin{itemize}
			\item chaque fonction a été testée
			\item chaque étape a été validée
		\end{itemize}
		\item \textbf{Organisation des informations à traiter}
		\begin{itemize}
			\item les données sont rangées dans des objets bien identifiés
		\end{itemize}
	\end{itemize}


	\item \textbf{PROGRAMMATION}
	\begin{itemize}
		\item \textbf{Ecriture globale du code et commentaires (PEP 8)}
		\begin{itemize}
			\item variables et fonctions respectant les conventions d'écriture standard
			\item commentaires utiles
		\end{itemize}		 
		\item \textbf{Utilisation, écriture de fonctions}
		\begin{itemize}
			\item paramètres et retours pertinents des fonctions
		\end{itemize}
		\item \textbf{Documentation des fonctions (PEP257)}
		\begin{itemize}
			\item paramètres et retours des fonctions sont documentés
		\end{itemize}
		\item Création de classes et d'objets
		\begin{itemize}
			\item classe contenant des attributs et méthodes pertinents
			\item aucune fonction n'est appelée en dehors d'un objet
		\end{itemize}
	\end{itemize}
	

	\item \textbf{INGENIEUR.E PHYSIQUE}
	\begin{itemize}
		\item \textbf{Graphiques pertinents et légendés}
		\begin{itemize}
			\item graphiques scientifiques (axes, titre...)
			\item axes des graphiques légendés (passage temps/fréquence)
		\end{itemize}		 
		\item \textbf{Organisation en actions élémentaires}
		\begin{itemize}
			\item les étapes sont découpées en fonctionnalité plus simple à tester
		\end{itemize}
		\item \textbf{Génération de données pertinentes de tests}
		\begin{itemize}
			\item choix de la position des sources pertinent
		\end{itemize}
		\item \textbf{Analyse des données et validation modèle}
		\begin{itemize}
			\item comparaison avec la théorie
			\item analyse pertinente des cartes obtenues
		\end{itemize}
	\end{itemize}
	
\medskip	
	
	\item \textbf{AVANCEMENT}
	\begin{itemize}
		\item Etapes 1 et 2 : x 0.5
		\item Etapes 1, 2 et 3 : x 0.7
		\item Une des ouvertures : x 1.0
	\end{itemize}
\end{itemize}



%----------------------------------------------------------------------------------------
\newpage

\hypertarget{stepbystep}{\heading{Quelques éléments supplémentaires}{6pt}}

\textbf{Modélisation d'une diode électroluminescente}

Les sources (par exemple des LEDs) seront modélisées de manière approchée (valable si l'on n'est pas trop près du composant) comme des \textbf{sources ponctuelles}. Ces sources ont un \textbf{diagramme de rayonnement} possédant une \textbf{symétrie de révolution} autour d'un axe orienté. 

\qquad

L'indicatrice de rayonnement pourra être considérée comme gaussienne, et caractérisée par son intensité visuelle vers l'avant sur l'axe $I_0$ (en candela) et sa largeur totale à mi-hauteur $\Delta$.

Cette indicatrice peut-être modélisée par l'équation suivante : $$I(\alpha) = I_0 \cdot \exp(-(4 \cdot \ln(2)) \cdot (\alpha/\Delta)^2)$$

où $\alpha$ est l'angle entre la direction d'émission et l'axe de la source ($\alpha \in [0^{\circ}, 180^{\circ}]$). 

\qquad 

\textbf{Positionnement d'une source}

Le positionnement de la source dans l'espace sera caractérisé par ses coordonnées $(x, y, z)$ et l'orientation de son axe de symétrie par deux angles ($\theta$ et $\phi$). 
	

\qquad 

\textbf{Eclairement / Formule de Bouguer}

L'éclairement fourni par une source ponctuelle en un point M de l'espace séparé d'une distance $d$ et d'une inclinaison de $\psi$ par rapport à la direction de la source ponctuelle, est données par la relation photométrique suivante : 

$$E = \frac{I \cdot \cos(\psi)}{d^2}$$

\qquad 

L'éclairement produit par N sources (incohérentes) est la somme des éclairements produits par chaque source.
%----------------------------------------------------------------------------------------

\end{document} 