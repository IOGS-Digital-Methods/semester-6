%----------------------------------------------------------------------------------------
%	DEVELIRABLES
%----------------------------------------------------------------------------------------

Pour valider cette session, vous devez \textbf{présenter} les \textbf{livrables suivants} lors de la séance 4 de ce bloc :

\begin{enumerate}
\item \textbf{Fonctions commentées} (selon la norme PEP 257) pour générer des signaux numériques appropriés
\item \textbf{Graphiques légendés} incluant toutes les données nécessaires à la bonne compréhension des données présentées : signal initial, transformée de Fourier du signal initial, signaux générés pour démoduler le signal, transformées de Fourier intermédiaires, signal démodulé
\item \textbf{Analyse des figures} en insistant sur la démarche ayant amené à la démodulation du signal
\item \textbf{BONUS : Fichiers démodulés} contenant les différents signaux démodulés
\end{enumerate}

\medskip

Ces livrables pourront prendre la forme d'un \textbf{compte-rendu} incluant une introduction à la problématique, les figures demandées ainsi que leur analyse.

Ce compte-rendu sera accompagné des \textbf{fichiers} \mbox{\textit{main.py}} et \mbox{\textit{signal\_processing.py}} contenant le programme principal permettant la génération des figures et de leurs légendes et les différentes fonctions commentées selon la norme PEP 257.

\textbf{Vous aurez 10 minutes lors de la séance 4 pour présenter l'ensemble de vos résultats et vos analyses.}