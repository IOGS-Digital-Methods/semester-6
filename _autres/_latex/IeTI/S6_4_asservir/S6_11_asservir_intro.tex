\begin{raggedleft}

\encadreObjectifs{
A la fin de cette thématique, les étudiant·e·s seront capables de :
	\begin{itemize}[label=$\bullet$,leftmargin=*]
		\item Schématiser une boucle d'asservissement
		\item Différencier les performances d'un système en boucle ouverte et en boucle fermée
		\item Rappeler le rôle d'un correcteur dans une boucle d'asservissement
	\end{itemize}
}


\encadreActivites{
\begin{itemize}[label=$\bullet$,leftmargin=*]
	\item Lectures (hors temps présentiel - en ligne)
	\begin{itemize}[label=$\triangleright$,leftmargin=*]
		\item TD2 du semestre 5 : Réaliser un étage de pré-amplification
		\item Fiche Résumé : Amplificateur Linéaire Intégré
	\end{itemize}
	\item Séance de \textbf{TD11}
	\item Séance de \textbf{TD12}
\end{itemize}
}

%\encadreCertification{
%La certification de ce module est disponible sur la plateforme eCampus, module 5N-027-SCI :
%\begin{itemize}[label=$\triangle$,leftmargin=*]
	%\item Test limité à \textbf{2 tentatives} / Durée d'environ \textbf{10 min} / Note minimale pour valider : \textbf{80\%}
%\end{itemize}
%
%{\scriptsize Un \textbf{test d'entrainement} (non limité) est également disponible sur la plateforme eCampus.}
%
%}

\encadreRessources{
\begin{itemize}[label=$\bullet$,leftmargin=*]
	\item Cours "Automatique" / Caroline Kulcsár - 2A Palaiseau
\end{itemize}
}

\end{raggedleft}