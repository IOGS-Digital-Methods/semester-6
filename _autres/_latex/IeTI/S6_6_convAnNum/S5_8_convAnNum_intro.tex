\begin{raggedleft}

\encadreObjectifs{
A la fin de cette thématique, les étudiant·e·s seront capables de :
\begin{itemize}[label=$\bullet$,leftmargin=*]
	\item énumérer les caractéristiques d'un Convertisseur Analogique-Numérique (CAN) et d'un Convertisseur Numerique-Analogique (CNA).
	\item analyser la structure d'un CNA
	\item évaluer les performances d'un CAN/CNA
\end{itemize}
}

\encadreActivites{
\begin{itemize}[label=$\bullet$,leftmargin=*]
	\item Lectures (hors temps présentiel - en ligne)
	\begin{itemize}[label=$\triangleright$,leftmargin=*]
		\item Cours : Codage des informations (J. VILLEMEJANE - 2013)
		\item Cours : Le Numérique et le binaire (H. BENISTY - 2016)
		\item Cours : La Conversion CAN et CNA (H. BENISTY - 2014)
	\end{itemize}
	\item Séance de \textbf{TD8}
	\item Séances de \textbf{TP3} et \textbf{TP4} (module TP CéTI)
\end{itemize}
}

%\encadreCertification{
%La certification de ce module est disponible sur la plateforme eCampus, module 5N-027-SCI :
%\begin{itemize}[label=$\triangle$,leftmargin=*]
	%\item Test limité à \textbf{2 tentatives} / Durée d'environ \textbf{10 min} / Note minimale pour valider : \textbf{80\%}
%\end{itemize}
%
%{\scriptsize Un \textbf{test d'entrainement} (non limité) est également disponible sur la plateforme eCampus.}
%
%}

\encadreRessources{
\begin{itemize}[label=$\bullet$,leftmargin=*]
	\item Introduction aux systèmes numériques - Julien Villemejane (2016)
	\item Monde numérique - Julien Villemejane (2013)
	\item Exercices supplémentaires proposés sur eCampus (avec correction)
\end{itemize}
}

\end{raggedleft}