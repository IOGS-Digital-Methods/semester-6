\begin{raggedleft}

\encadreObjectifs{
A la fin de cette thématique, les étudiant·e·s seront capables de :
\begin{itemize}[label=$\bullet$,leftmargin=*]
	\item Modéliser un montage de photodétection de type transimpédance
\end{itemize}
}

\encadreActivites{
\begin{itemize}[label=$\bullet$,leftmargin=*]
	\item Lectures (hors temps présentiel - en ligne)
	\begin{itemize}[label=$\triangleright$,leftmargin=*]
		\item TD7 du Semestre 5 : Détecter des photons
		\item Fiche résumé : Analyse Harmonique / Ordre 1
		\item Fiche résumé : Analyse Harmonique / Ordre 2	
		\item Fiche résumé : Photodétection
	\end{itemize}
	\item Séance de \textbf{TD10}
	\item Séances de TP (module TP CéTI)
\end{itemize}
}

%\encadreCertification{
%La certification de ce module est disponible sur la plateforme eCampus, module 5N-027-SCI :
%\begin{itemize}[label=$\triangle$,leftmargin=*]
	%\item Test limité à \textbf{2 tentatives} / Durée d'environ \textbf{10 min} / Note minimale pour valider : \textbf{80\%}
%\end{itemize}
%
%{\scriptsize Un \textbf{test d'entrainement} (non limité) est également disponible sur la plateforme eCampus.}
%
%}

\encadreRessources{
\begin{itemize}[label=$\bullet$,leftmargin=*]
	\item Modélisations sous MATLAB ( http://lense.institutoptique.fr/simuler/ ) :
	\begin{itemize}[label=$\triangleright$,leftmargin=*]
		\item Photodétection : Comparaison entre système de photodétection simple et transimpédance
		\item Photodétection : Montage transimpédance / Réponse en fréquence
		\item Photodétection : Montage transimpédance / Comparaison
	\end{itemize}	
\end{itemize}
}

\end{raggedleft}