%----------------------------------------------------------------------------------------
%	SHORT INTRODUCTION
%----------------------------------------------------------------------------------------

Dans le cadre du module \textbf{ONIP-2}, vous serez amenés à découvrir les concepts de la \textbf{programmation orientée objet} et à développer un  mini-projet selon ces nouvelles règles.

Dans le projet proposé, on cherchera à \textbf{calculer la carte d'éclairement} produit par un \textbf{ensemble de sources incohérentes}.

Les sources seront modélisées de manière approchée (valable si l'on n'est pas trop près du composant) comme une \textbf{source ponctuelle} ayant un diagramme de rayonnement possédant une symétrie de révolution autour d'un axe.

\textit{Aucune fonction ne devra être utilisée en dehors d'un objet.}